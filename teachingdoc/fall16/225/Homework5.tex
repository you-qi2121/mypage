\documentclass[11pt]{article}
\usepackage[all,2cell]{xy} \UseAllTwocells \SilentMatrices
\usepackage{latexsym,amsfonts,amssymb}
\usepackage{amsmath,amsthm,amscd}
\usepackage{hyperref,psfrag}
%\usepackage{diagcat}
\usepackage{color}
\usepackage{etoolbox}

\usepackage[dvips]{epsfig}

\usepackage{psfrag}
\def\drawing#1{\begin{center}\epsfig{file=#1}\end{center}}
% including eps files

\usepackage{graphicx}

\usepackage{a4wide}
\renewcommand*\rmdefault{ppl}\normalfont\upshape

%\usepackage{fancyhdr}
%\pagestyle{fancyplain}
%\renewcommand{\sectionmark}[1]{\markboth{#1}{}}
%\renewcommand{\subsectionmark}[1]{\markright{#1}}
%\lhead[\fancyplain{}{\bfseries\thepage}]{\fancyplain{}{\sl\bfseries\rightmark}}
%\rhead[\fancyplain{}{\sl\bfseries\leftmark}]{\fancyplain{}{\bfseries\thepage}}
%\cfoot{}

\hfuzz=6pc


%------------------------------------------------------------------------------%
\theoremstyle{plain}
\newtheorem{thm}{Theorem}
\newtheorem{prop}[thm]{Proposition}
\newtheorem{lemma}[thm]{Lemma}
\newtheorem{cor}[thm]{Corollary}
\newtheorem{conj}[thm]{Conjecture}

\theoremstyle{definition}

\newtheorem{defn}[thm]{Definition}
\newtheorem{eg}[thm]{Example}
\newtheorem{rmk}[thm]{Remark}
\newtheorem{ntn}[thm]{Notation}
\newtheorem{ex}[thm]{Exercise}

\usepackage{bbm}
\def\1{\mathbbm 1}
\def\R{\mathbb R}
\def\Q{\mathbb Q}
\def\Z{\mathbb Z}
\def\N{\mathbb N}
\def\C{\mathbb C}
\def\F{\mathbb F}
\def\P{\mathbb P}
\def\o{\otimes}
\def\lra{\longrightarrow}
\def\zb{\overline{z}}
\def\Arg{\mathrm{Arg}}
\def\Log{\mathrm{Log}}

\begin{document}
\begin{center}
{\Large \bf Exercises for Week 5}
\end{center}
The work handed in should be entirely your own. You can consult the textbook and/or the class notes but nothing else. To receive full credit, justify your answer in a clear and logical way. Due Oct.~6.

\paragraph{Reading.} Read Sections 2.3-2.4 of the textbook carefully.

\begin{enumerate}
\item Section 2.3 Exercise 1, 3, 12, 13, 15, 18
\item Follow the steps and prove Theorem 2.14 on your own.

We have known that, if $V$ and $ W$ are vector spaces, and $\beta=\{v_1,\dots, v_n\}$ and $\gamma=\{w_1,\dots, w_m\}$ are ordered bases for $V$ and $W$ respectively. Then, via the chosen bases, vectors can be expanded into a (column) of numbers:
\[
v\in V \Rightarrow v=\sum_{i=1}^n a_i v_i \Rightarrow 
\left(
\begin{matrix}
a_1\\
\vdots\\
a_n
\end{matrix}
\right)\in \F^n.
\] 
Since these tuple of numbers depend on the choice of $\beta$, we denote the column vector by
\[
[v]^{\beta}:=
\left(
\begin{matrix}
a_1\\
\vdots\\
a_n
\end{matrix}
\right).
\]
(The text book uses $[v]_{\beta}$ instead which is a bit awkward from what you'll show)
We have also learnt in class that, any linear map $T:V\lra W$ is completely determined by its matrix $A=[T]_\beta^\gamma$ with respect to $\beta$ and $\gamma$. Then we have
\[
[T(v)]^{\gamma}=[T]_{\beta}^{\gamma}\cdot [v]^{\beta}. \quad \quad (*)
\]
\begin{enumerate}
\item[(1)] Define the following linear map, for a fixed vector $v\in V$ by
\[
F_v: \F \lra V,\quad F_v(a)=av.
\]
Show that $F_v$ is linear (state which axioms of vector spaces you use in the proof).
\item[(2)] $\F$ as a 1-dimensional space over $\F$ has the standard basis $\alpha:=\{1\}$. Compute the matrix $[F_v]_{\alpha}^\beta$.
\item[(3)] Now use the compostion of linear maps giving rise to matrix multiplication to show that $(*)$ is true.
\end{enumerate}


\item Section 2.4 Exercise 1, 2 (a) (d) (e), 3 (c), (d), 4, 6, 9, 12, 19, 22 
\end{enumerate}
\end{document} 