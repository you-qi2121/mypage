\documentclass[11pt]{article}
\usepackage[all,2cell]{xy} \UseAllTwocells \SilentMatrices
\usepackage{latexsym,amsfonts,amssymb}
\usepackage{amsmath,amsthm,amscd}
\usepackage{hyperref,psfrag}
%\usepackage{diagcat}
\usepackage{color}
\usepackage{etoolbox}

\usepackage[dvips]{epsfig}

\usepackage{psfrag}
\def\drawing#1{\begin{center}\epsfig{file=#1}\end{center}}
% including eps files

\usepackage{graphicx}

\usepackage{a4wide}
\renewcommand*\rmdefault{ppl}\normalfont\upshape

%\usepackage{fancyhdr}
%\pagestyle{fancyplain}
%\renewcommand{\sectionmark}[1]{\markboth{#1}{}}
%\renewcommand{\subsectionmark}[1]{\markright{#1}}
%\lhead[\fancyplain{}{\bfseries\thepage}]{\fancyplain{}{\sl\bfseries\rightmark}}
%\rhead[\fancyplain{}{\sl\bfseries\leftmark}]{\fancyplain{}{\bfseries\thepage}}
%\cfoot{}

\hfuzz=6pc


%------------------------------------------------------------------------------%
\theoremstyle{plain}
\newtheorem{thm}{Theorem}
\newtheorem{prop}[thm]{Proposition}
\newtheorem{lemma}[thm]{Lemma}
\newtheorem{cor}[thm]{Corollary}
\newtheorem{conj}[thm]{Conjecture}

\theoremstyle{definition}

\newtheorem{defn}[thm]{Definition}
\newtheorem{eg}[thm]{Example}
\newtheorem{rmk}[thm]{Remark}
\newtheorem{ntn}[thm]{Notation}
\newtheorem{ex}[thm]{Exercise}

\usepackage{bbm}
\def\1{\mathbbm 1}
\def\R{\mathbb R}
\def\Q{\mathbb Q}
\def\Z{\mathbb Z}
\def\N{\mathbb N}
\def\C{\mathbb C}
\def\F{\mathbb F}
\def\P{\mathbb P}
\def\o{\otimes}
\def\lra{\longrightarrow}
\def\zb{\overline{z}}
\def\Arg{\mathrm{Arg}}
\def\Log{\mathrm{Log}}

\begin{document}
\begin{center}
{\Large \bf Exercises for Week 7}
\end{center}
The work handed in should be entirely your own. You can consult the textbook and/or the class notes but nothing else. To receive full credit, justify your answer in a clear and logical way. Due Oct. 27th.

\paragraph{Reading.} Read Sections 3.2--3.4 of the textbook carefully.

\begin{enumerate}
\item Section 3.1 Exercise 1, 2, 5
\item Section 3.2 Exercise 1, 2 (a) (d) (e), 3, 5 (b), (c), (f), 7, 15, 18, 21.
\item Section 3.3 Exercise 1, 2 (a) (c) (e), 5, 6, 8
\item Prove the following statement.

In class, we discussed the rank of post-composing with an isomorphism. In this exercise, show that the nullity doesn't change either by precomposing with an isomorphism.

Let $T:U\lra V$ be a linear map over $\F$ and $S:U^\prime \lra U$ be an isomorphism. Show that $S$ restricts to an isomorphism
\[
S: \mathrm{Ker}(T\circ S) \lra \mathrm{Ker}(T).
\]

In particular, $n(T)=n(T\circ S)$.

\item Section 3.4 Exercise 1, 2 (a) (d) (h), 3
\end{enumerate}
\end{document} 