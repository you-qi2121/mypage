\documentclass[margin,line]{res}
\usepackage{url}

\oddsidemargin -.5in \evensidemargin -.5in \textwidth=6.0in
\itemsep=0in
\parsep=0in

\newenvironment{list1}{
  \begin{list}{\ding{113}}{%
      \setlength{\itemsep}{0in}
      \setlength{\parsep}{0in} \setlength{\parskip}{0in}
      \setlength{\topsep}{0in} \setlength{\partopsep}{0in}
      \setlength{\leftmargin}{0.17in}}}{\end{list}}
\newenvironment{list2}{
  \begin{list}{$\bullet$}{%
      \setlength{\itemsep}{0in}
      \setlength{\parsep}{0in} \setlength{\parskip}{0in}
      \setlength{\topsep}{0in} \setlength{\partopsep}{0in}
      \setlength{\leftmargin}{0.2in}}}{\end{list}}


\begin{document}

\name{Syllabus of Linear Algebra and Matrix Theory \vspace*{.1in}}

\begin{resume}

\section{\sc Key Information}
\vspace{.05in}
\begin{tabular}{@{}p{3.4in}p{4in}}
 Instructor: You Qi            & Tel: (203) 432-6859   \\
Time: Every Tue, Thu 9:00--10:15 am & Venue: 215 LOM\\
Office: 404 DL, 10 Hillhouse   & E-mail:  you.qi@yale.edu \\
Office Hour: Tue, Thu, 10:30--11:30 am &\\
& \\
Teaching Assistant: Andrei Florin Deneanu & E-mail: andreiflorin.deneanu@yale.edu\\
Discussion Session: Every Tue, 6:00--7:00pm & Venue: AKW 100\\
TA Office Hour: Every Wed, 9:00--10:30am  & Venue: AKW 114 \\
& \\
Midterm Date and Venue: Oct.~11th & In class\\
Final Date: Dec.~19 & TBA
\end{tabular}
Course webpage: \url{http://math.yale.edu/~yq64/fall2016225}.

\section{\sc Textbook}
We will mostly follow \emph{Linear Algebra, 4th Edition}
by Stephen H. Friedberg, Arnold J. Insel, Lawrence E. Spence. Pearson, 2003.

\section{\sc Prerequisites}
A good understanding of 3-dimensional Euclidean geometry. Functions. Polynomial functions in one variable. It helps to have a basic understanding of calculus (or to be taking it concurrently).


\section{\sc Content}
Our basic goal is to have a basic understanding of the important subject of linear algebra. This is the mathematical tool you use after simplifying scientific models by their linear approximation. Our plan is to cover the essential part of the six chapters of the textbook, as well as some additional topics if time permits. Of course we'll adjust the speed of teaching as we proceed along. Here is a list of sections of the textbook that we will cover, which we will be going through at the speed of about 1.5 sections per class.
\begin{list2}
\item Chapter 1, 1.1--1.6.
\item Chapter 2, 2.1--2.5, 2.7.
\item Chapter 3, 3.1--3.4.
\item Chapter 4, 4.1--4.3.
\item Chapter 5, 5.1--5.3.
\item Chapter 6, 6.1--6.5.
\end{list2}

A more applied/computational oriented student should also try out Math 222 during the shopping period.

\section{\sc Advice}
{\bf You are required to attend all the lectures}. Since our
lectures may differ from the text book, and the
schedule might change in occasion. As a general principle for
taking math courses, \emph{take twice the amount of time of lectures
to review what you learnt in class, and do a lot of exercises!} What
we hope to achieve is not only the knowledge but also the ability
to think logically and independently. Feel free to let me know if
some points are unclear to you and ask for more explanations. Any
suggestions about the teaching will be warmly welcomed.

In case you have a conflict with any of the exams, please contact the
instructor as soon as possible and at least two weeks before the
exam. I will schedule a make up exam for you in my office. In the case of a fire alarm or a similar emergency evacuation during either of the midterms or the final exam, leave your exams in the room, face down, before evacuating. Under no circumstances should you take the exam with you.


\section{\sc Assignments}
Homeworks will be assigned each Tuesday, and due the Tuesday a week after. No late homeworks will be accepted. Discussing the problems with other students is encouraged, but each student must write solutions on his/her own. Quizzes and exams will have similar problems as the examples we
do in class and exercises you will do after class. As another general principle in math,
\emph{practice makes perfect}.

\section{\sc Grading}
Your final grade will be based on weekly assignments, a midterm exam and a final exam. We will de-emphasize counting homeworks towards your final grade, since this is the place you are allowed to make mistakes, and get them clarified.
\\

\begin{list2}
\item Homeworks, 10 \%
\item Quizzes, 20 \%
\item Mid-term, 30 \%
\item Final, 40 \%
\end{list2}

\section{\sc Academic Honesty} Any form of
cheating during the midterm or the final will result in you failing the
course and the matter being reported to your dean.

\end{resume}

\end{document}



