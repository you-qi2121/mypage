\documentclass[11pt]{article}
\usepackage[all,2cell]{xy} \UseAllTwocells \SilentMatrices
\usepackage{latexsym,amsfonts,amssymb}
\usepackage{amsmath,amsthm,amscd}
\usepackage{hyperref,psfrag}
%\usepackage{diagcat}
\usepackage{color}
\usepackage{etoolbox}

\usepackage[dvips]{epsfig}

\usepackage{psfrag}
\def\drawing#1{\begin{center}\epsfig{file=#1}\end{center}}
% including eps files

\usepackage{graphicx}

\usepackage{a4wide}
\renewcommand*\rmdefault{ppl}\normalfont\upshape

%\usepackage{fancyhdr}
%\pagestyle{fancyplain}
%\renewcommand{\sectionmark}[1]{\markboth{#1}{}}
%\renewcommand{\subsectionmark}[1]{\markright{#1}}
%\lhead[\fancyplain{}{\bfseries\thepage}]{\fancyplain{}{\sl\bfseries\rightmark}}
%\rhead[\fancyplain{}{\sl\bfseries\leftmark}]{\fancyplain{}{\bfseries\thepage}}
%\cfoot{}

\hfuzz=6pc


%------------------------------------------------------------------------------%
\theoremstyle{plain}
\newtheorem{thm}{Theorem}
\newtheorem{prop}[thm]{Proposition}
\newtheorem{lemma}[thm]{Lemma}
\newtheorem{cor}[thm]{Corollary}
\newtheorem{conj}[thm]{Conjecture}

\theoremstyle{definition}

\newtheorem{defn}[thm]{Definition}
\newtheorem{eg}[thm]{Example}
\newtheorem{rmk}[thm]{Remark}
\newtheorem{ntn}[thm]{Notation}
\newtheorem{ex}[thm]{Exercise}

\usepackage{bbm}
\def\1{\mathbbm 1}
\def\R{\mathbb R}
\def\Q{\mathbb Q}
\def\Z{\mathbb Z}
\def\N{\mathbb N}
\def\C{\mathbb C}
\def\F{\mathbb F}
\def\P{\mathbb P}
\def\o{\otimes}
\def\lra{\longrightarrow}
\def\zb{\overline{z}}
\def\Arg{\mathrm{Arg}}
\def\Log{\mathrm{Log}}

\begin{document}
\begin{center}
{\Large \bf Exercises for Week 9}
\end{center}
The work handed in should be entirely your own. You can consult Gamelin and/or the class notes but nothing else. To receive full credit, justify your answer in a clear and logical way. Due Apr. 10.

\paragraph{Reading.} Read Chapter V of the textbook carefully (better before you attempt the homework problems).
\begin{enumerate}

\item (a) Prove the following theorem rigorously using $\epsilon$-$\delta$ language.

\begin{thm}Let $\{f_j\}$ be a sequence of complex-valued functions defined on a subset $E$ of the complex plane. If each $f_j$ is continuous on $E$, and the sequence of functions converge uniformly to $f$ on $E$, then $f$ is also continuous on $E$.
\end{thm}

(b) Find an example of a sequence of functions on the unit interval $(-1,1]\subset \R$ that converges pointwise, but the limit function is not continuous.

\item Section V.3. Exercises 3, 4.

\item Section V.4. Exercises 1 (b), (d), (f), 6, 11, 13.

\item Section V.5. Exercises 1 (a), (c), 4.

\end{enumerate}
\end{document} 