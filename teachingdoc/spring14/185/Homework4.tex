\documentclass[11pt]{article}
\usepackage[all,2cell]{xy} \UseAllTwocells \SilentMatrices
\usepackage{latexsym,amsfonts,amssymb}
\usepackage{amsmath,amsthm,amscd}
\usepackage{hyperref,psfrag}
%\usepackage{diagcat}
\usepackage{color}
\usepackage{etoolbox}

\usepackage[dvips]{epsfig}

\usepackage{psfrag}
\def\drawing#1{\begin{center}\epsfig{file=#1}\end{center}}
% including eps files

\usepackage{graphicx}

\usepackage{a4wide}
\renewcommand*\rmdefault{ppl}\normalfont\upshape

%\usepackage{fancyhdr}
%\pagestyle{fancyplain}
%\renewcommand{\sectionmark}[1]{\markboth{#1}{}}
%\renewcommand{\subsectionmark}[1]{\markright{#1}}
%\lhead[\fancyplain{}{\bfseries\thepage}]{\fancyplain{}{\sl\bfseries\rightmark}}
%\rhead[\fancyplain{}{\sl\bfseries\leftmark}]{\fancyplain{}{\bfseries\thepage}}
%\cfoot{}

\hfuzz=6pc


%------------------------------------------------------------------------------%
\theoremstyle{plain}
\newtheorem{thm}{Theorem}
\newtheorem{prop}[thm]{Proposition}
\newtheorem{lemma}[thm]{Lemma}
\newtheorem{cor}[thm]{Corollary}
\newtheorem{conj}[thm]{Conjecture}

\theoremstyle{definition}

\newtheorem{defn}[thm]{Definition}
\newtheorem{eg}[thm]{Example}
\newtheorem{rmk}[thm]{Remark}
\newtheorem{ntn}[thm]{Notation}
\newtheorem{ex}[thm]{Exercise}

\usepackage{bbm}
\def\1{\mathbbm 1}
\def\R{\mathbb R}
\def\Q{\mathbb Q}
\def\Z{\mathbb Z}
\def\N{\mathbb N}
\def\C{\mathbb C}
\def\F{\mathbb F}
\def\P{\mathbb P}
\def\o{\otimes}
\def\lra{\longrightarrow}
\def\zb{\overline{z}}
\def\Arg{\mathrm{Arg}}
\def\Log{\mathrm{Log}}

\begin{document}
\begin{center}
{\Large \bf Exercises for Week 4}
\end{center}
The work handed in should be entirely your own. You can consult Gamelin and/or the class notes but nothing else. To receive full credit, justify your answer in a clear and logical way. Due Feb.~20.

\paragraph{Reading.} Read Sections 2.3-2.5 of the textbook carefully (better before you attempt the homework problems). If you have forgotten about line integrals and Green's theorem, there is a nice summary in the textbook in Section 3.1. Alternatively, consult your calculus textbook.

\begin{enumerate}
\item Define the complex sine and cosine functions by
\[
\sin z :=\frac{e^{iz}-e^{-iz}}{2i}, \quad \cos z:=\frac{e^{iz}+e^{-iz}}{2}.
\]
(See the textbook 1.8 for more properties about them.) Identify their real and imaginary parts $u$ and $v$, and show that they are complex analytic by checking the Cauchy-Riemann equation hold for $u$ and $v$. Use our theorem to identify their complex derivative, and compare with the usual real case.

\item Section II.3. Exercises 6, 8.

\item Section II.4 Exercises 3, 7.

\item Section II.5 Exercises 1 (a),  (d), 3, 5.

\end{enumerate}
\end{document} 