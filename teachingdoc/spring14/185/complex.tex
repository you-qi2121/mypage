\documentclass[margin,line]{res}
\usepackage{url}

\oddsidemargin -.5in \evensidemargin -.5in \textwidth=6.0in
\itemsep=0in
\parsep=0in

\newenvironment{list1}{
  \begin{list}{\ding{113}}{%
      \setlength{\itemsep}{0in}
      \setlength{\parsep}{0in} \setlength{\parskip}{0in}
      \setlength{\topsep}{0in} \setlength{\partopsep}{0in}
      \setlength{\leftmargin}{0.17in}}}{\end{list}}
\newenvironment{list2}{
  \begin{list}{$\bullet$}{%
      \setlength{\itemsep}{0in}
      \setlength{\parsep}{0in} \setlength{\parskip}{0in}
      \setlength{\topsep}{0in} \setlength{\partopsep}{0in}
      \setlength{\leftmargin}{0.2in}}}{\end{list}}


\begin{document}

\name{Syllabus of Introduction to Complex Analysis \vspace*{.1in}}

\begin{resume}

\section{\sc Related Information}
\vspace{.05in}
\begin{tabular}{@{}p{3.4in}p{4in}}
 Instructor: You Qi            &   \\
Time: Every Tue, Thu 8:00-9:30am & Venue: 200 Wheeler\\
Office: Room 849, Evans   & E-mail:  yq2121@berkeley.edu \\
Office Hour: Tue, Thu, 9:30-10:30 am & Fri. 1:30-2:30am (tentative)\\
Midterm Date: Mar 20th, in class. &\\
Final Date: May 15th, 7-10 pm.
\end{tabular}
Course webpage: \url{http://math.berkeley.edu/~yq2121/spring2014185}.

\section{\sc Textbook}
We will mostly follow \emph{Complex Analysis}
by T. W. Gamelin. Springer. 2003.

\section{\sc Prerequisites}
A good understanding of calculus and linear algebra, and some basic familiarity with $\epsilon$-$\delta$ language.


\section{\sc Content}
Our basic goal is to cover some basic properties of complex analytic functions, and develop the analogues of differential and integral calculus over complex numbers. Our plan is to cover the essential part of the first seven chapters of the textbook, as well as some topics from the second part if time permits. Of course we'll adjust the speed of teaching as we proceed along. Here is the basic topics that we will go through.
\begin{list2}
\item Elementary geometry of complex numbers, some basic complex functions.
\item Analytic functions, harmonic functions.
\item Cauchy integral formula and applications.
\item Power series, Laurent series, and singularities.
\item Residues, logarithmic integral.
\end{list2}

\section{\sc Advice}
{\bf You are required to attend all the lectures}. Since our
lectures may differ from the text book, and the
schedule might change in occasion. As a general principle for
taking math courses, \emph{take twice the amount of time of lectures
to review what you learnt in class, and do a lot of exercises!} What
we hope to achieve is not only the knowledge but also the ability
to think logically and independently. Feel free to let me know if
some points are unclear to you and ask for more explanations. Any
suggestions about the teaching will be warmly welcomed.

In case you have a conflict with any of the exams, please contact the
instructor as soon as possible and at least two weeks before the
exam. I will schedule a make up exam for you in my office. In the case of a fire alarm during either of the midterms or the final exam, leave your exams in the room, face down, before evacuating. Under no circumstances should you take the exam with you.


\section{\sc Assignments}
Homeworks will be assigned each Thursday, and due the Thursday a week after. No late homeworks will be accepted. Discussing the problems with other students is encouraged, but each student must write solutions on his/her own. Quizzes and exams will have similar problems as the examples we
do in class and exercises you will do after class. As another general principle in math,
\emph{practice makes perfect}.

\section{\sc Grading}
\begin{list2}
\item Homeworks, 5 \%
\item Quizzes, 25 \%
\item Mid-term, 30 \%
\item Final, 40 \%
\end{list2}

\section{\sc Academic Honesty} Any form of
cheating during midterm or final will result in you failing the
course and the matter being reported to your dean.

\end{resume}

\end{document}



