\documentclass[11pt]{article}
\usepackage[all,2cell]{xy} \UseAllTwocells \SilentMatrices
\usepackage{latexsym,amsfonts,amssymb}
\usepackage{amsmath,amsthm,amscd}
\usepackage{hyperref,psfrag}
%\usepackage{diagcat}
\usepackage{color}
\usepackage{etoolbox}

\usepackage[dvips]{epsfig}

\usepackage{psfrag}
\def\drawing#1{\begin{center}\epsfig{file=#1}\end{center}}
% including eps files

\usepackage{graphicx}

\usepackage{a4wide}
\renewcommand*\rmdefault{ppl}\normalfont\upshape

%\usepackage{fancyhdr}
%\pagestyle{fancyplain}
%\renewcommand{\sectionmark}[1]{\markboth{#1}{}}
%\renewcommand{\subsectionmark}[1]{\markright{#1}}
%\lhead[\fancyplain{}{\bfseries\thepage}]{\fancyplain{}{\sl\bfseries\rightmark}}
%\rhead[\fancyplain{}{\sl\bfseries\leftmark}]{\fancyplain{}{\bfseries\thepage}}
%\cfoot{}

\hfuzz=6pc


%------------------------------------------------------------------------------%
\theoremstyle{plain}
\newtheorem{thm}{Theorem}
\newtheorem{prop}[thm]{Proposition}
\newtheorem{lemma}[thm]{Lemma}
\newtheorem{cor}[thm]{Corollary}
\newtheorem{conj}[thm]{Conjecture}

\theoremstyle{definition}

\newtheorem{defn}[thm]{Definition}
\newtheorem{eg}[thm]{Example}
\newtheorem{rmk}[thm]{Remark}
\newtheorem{ntn}[thm]{Notation}
\newtheorem{ex}[thm]{Exercise}

\usepackage{bbm}
\def\1{\mathbbm 1}
\def\R{\mathbb R}
\def\Q{\mathbb Q}
\def\Z{\mathbb Z}
\def\N{\mathbb N}
\def\C{\mathbb C}
\def\F{\mathbb F}
\def\P{\mathbb P}
\def\o{\otimes}
\def\lra{\longrightarrow}
\def\zb{\overline{z}}
\def\Arg{\mathrm{Arg}}
\def\Log{\mathrm{Log}}

\begin{document}
\begin{center}
{\Large \bf Problem Set for Week 8}
\end{center}
The work handed in should be entirely your own. You can consult Stewart and/or the class notes but nothing else. To receive full credit, justify your answer in a clear and logical way. Due March 25.

\paragraph{Reading.} This is the most important part of the homework: Read Sections 16.3--16.4 of the textbook carefully. Since we'll be having the second midterm soon after the Spring break, it is a good idea to start reviewing what we have covered from Chapter 15.


\begin{enumerate}
\item Section 16.4 Exercises 2, 10, 14, 21, 27.
\item Evaluate the integral
$\int_{C}\mathbf{F} \cdot d\mathbf{r}$.
Here $\mathbf{F}$ is the field
\[
\mathbf{F} =(\sin(x^2) + y)\mathbf{i} + (xy^2 + y^4)\mathbf{j},
\]
and the curve $C$ goes from $(0, -1)$ to $(0, 1)$ along the parabola $x = 1 - y^2$.
\item Evaluate the integral
$\oint_{C}\mathbf{F}\cdot  d\mathbf{r}$
Here $\mathbf{F}$ is the field
\[
\mathbf{F} = \frac{-y}{x^2 + y^2}
\mathbf{i} +(\frac{x}{x^2 + y^2}+ x)\mathbf{j}.
\]
The closed curve $C$ goes from $(-4, -12)$ to $(4, -12)$ on a straight line, then back to $(-4, -12)$ along
the parabola $y = 4 - x^2$.
Note that $\mathbf{F}$ is not defined at the origin.
\end{enumerate}
\end{document} 