\documentclass[11pt]{article}
\usepackage[all,2cell]{xy} \UseAllTwocells \SilentMatrices
\usepackage{latexsym,amsfonts,amssymb}
\usepackage{amsmath,amsthm,amscd}
\usepackage{hyperref,psfrag}
%\usepackage{diagcat}
\usepackage{color}
\usepackage{etoolbox}

\usepackage[dvips]{epsfig}

\usepackage{psfrag}
\def\drawing#1{\begin{center}\epsfig{file=#1}\end{center}}
% including eps files

\usepackage{graphicx}

\usepackage{a4wide}
\renewcommand*\rmdefault{ppl}\normalfont\upshape

%\usepackage{fancyhdr}
%\pagestyle{fancyplain}
%\renewcommand{\sectionmark}[1]{\markboth{#1}{}}
%\renewcommand{\subsectionmark}[1]{\markright{#1}}
%\lhead[\fancyplain{}{\bfseries\thepage}]{\fancyplain{}{\sl\bfseries\rightmark}}
%\rhead[\fancyplain{}{\sl\bfseries\leftmark}]{\fancyplain{}{\bfseries\thepage}}
%\cfoot{}

\hfuzz=6pc


%------------------------------------------------------------------------------%
\theoremstyle{plain}
\newtheorem{thm}{Theorem}
\newtheorem{prop}[thm]{Proposition}
\newtheorem{lemma}[thm]{Lemma}
\newtheorem{cor}[thm]{Corollary}
\newtheorem{conj}[thm]{Conjecture}

\theoremstyle{definition}

\newtheorem{defn}[thm]{Definition}
\newtheorem{eg}[thm]{Example}
\newtheorem{rmk}[thm]{Remark}
\newtheorem{ntn}[thm]{Notation}
\newtheorem{ex}[thm]{Exercise}

\usepackage{bbm}
\def\R{\mathbb R}
\def\Q{\mathbb Q}
\def\Z{\mathbb Z}
\def\N{\mathbb N}
\def\C{\mathbb C}
%\def\F{\mathbb F}
\def\P{\mathbb P}
\def\a{\mathbf a}
\def\b{\mathbf b}
\def\c{\mathbf c}
\def\r{\mathbf r}
\def\s{\mathbf s}
\def\t{\mathbf t}
\def\u{\mathbf u}
\def\v{\mathbf v}
\def\w{\mathbf w}
\def\i{\mathbf i}
\def\j{\mathbf j}
\def\k{\mathbf k}
\def\n{\mathbf n}
\def\x{\mathbf x}
\def\y{\mathbf y}
\def\z{\mathbf z}
\def\B{\mathbf B}
\def\E{\mathbf E}
\def\F{\mathbf F}
\def\S{\mathbf S}
\def\G{\mathbf G}
\def\o{\otimes}
\def\lra{\longrightarrow}
\def\zb{\overline{z}}

\begin{document}
\begin{center}
{\Large \bf Problem Set for Week 12--13}
\end{center}
The work handed in should be entirely your own. You can consult Stewart and/or the class notes but nothing else. To receive full credit, justify your answer in a clear and logical way. Due Friday, Apr. 24.

\paragraph{Reading.} This is the most important part of the homework: Read Sections 15.8--15.9, 16.9 of the textbook carefully.

\begin{enumerate}
\item Section 15.8 Exercises 8, 10, 16, 18, 24, 30,
\item Section 15.9 Exercises 8, 10, 14, 22, 26, 40,
\item The solid $B$ is the cap of a sphere, given by the equations
$x^2 + y^2 + z^2 \leq 4$, $z \geq 1$. Evaluate the integral $\iiint_BzdV$

a) in cylindrical coordinates,

b) in spherical coordinates.
\item Section 16.9 Exercises 4, 8, 12, 18, 26,
\item Evaluate the integral
$\iint_S
\F \cdot d\S$.
Here $\F$ is the field
\[
\F(x,y,z) =\frac{x^2}{2}\i + y \j + \k,
\]
and the surface $S$ is the top of a cone, oriented upward, given by the equations
\[
z = 1 -\sqrt{x^2 + y^2}, \quad z = 0.
\]
\item Evaluate the integral $\iint_S\F \cdot d\S$. Here $\F$ is the field
\[
\F(x,y,z) =\frac{x}{(x^2 + y^2 + z^2)^{3/2}}\i +
\frac{y}{(x^2 + y^2 + z^2)^{3/2}}\j+
\frac{z}{(x^2 + y^2 + z^2)^{3/2}}\k,
\]
and the surface $S$ is the ellipsoid, oriented outward, given by the equation
\[
\frac{x^2}{4}+\frac{y^2}{9}+\frac{z^2}{16} = 1.
\]
\end{enumerate}
\end{document} 