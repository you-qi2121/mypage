\documentclass[12pt]{article}
\title{}
\date{}
\usepackage{amsmath,amsfonts,amsthm,amssymb,mathrsfs}
\usepackage{times}
%\usepackage{hyperref}
%\usepackage{CJKutf8}
\textwidth=16cm
\textheight=23cm
\voffset=-1.8cm
\hoffset=-1.1cm
%------------------------------------------------------------------------------%
\theoremstyle{plain}
\newtheorem{thm}{Theorem}[section]
\newtheorem{lem}[thm]{Lemma}
\newtheorem{cor}[thm]{Corollary}
\theoremstyle{definition}
\newtheorem{defn}{Definition}[section]
\theoremstyle{remark}
\newtheorem*{remark}{Remark}
\newtheorem*{Ex}{Example}
\newtheorem*{ex}{Exercise}

\newcommand{\ba}{\beta}
\newcommand{\da}{\delta}
\newcommand{\Da}{\Delta}
\newcommand{\ta}{\theta}
\newcommand{\Ta}{\Theta}
\newcommand{\la}{\lambda}
\newcommand{\Oa}{\Omega}
\newcommand{\vphi}{\varphi}
\newcommand{\N}{\mathbb{N}}
\newcommand{\Z}{\mathbb{Z}}
\newcommand{\R}{\mathbb{R}}
\newcommand{\C}{\mathbb{C}}
\newcommand{\lra}{\longrightarrow}

\newcommand{\pl}{\partial}
\newcommand{\fc}{\frac}
\newcommand{\na}{\nabla}
\newcommand{\ol}{\overline}
\newcommand{\lt}{\left}
\newcommand{\rt}{\right}
\newcommand{\rw}{\rightarrow}

\newcommand{\mf}{\mathbf}
\newcommand{\mb}{\mathbb}
\newcommand{\ml}{\mathcal}

\newcommand{\bs}{\boldsymbol}
\newcommand{\wt}{\widetilde}

\pagestyle{empty}

\begin{document}
\begin{center}
{\Large \bf Exercises for Week 1 and 2}
\end{center}
The work handed in should be entirely your own. You can consult any abstract algebra textbook (e.g.~Dummit and Foote, Artin) and/or the class notes but nothing else. To receive full credit, justify your answer in a clear and logical way. Due Feb. 2.

\begin{enumerate}
\item Let $G:=\{a+b\sqrt{2}|(a,b)\in \mathbb{Q}^2\backslash (0,0)\}$. Show that $G$ is a group under multiplication.

\item Prove that if all elements of a group $G$ satisfy $x^2=1$, then $G$ is abelian.

\item List all subgroups of $S_3$.

\item Let $G$ be a group and $g_0$ be a fixed element of $G$. Show that the set $Z_G(g_0):=\{g\in G|gg_0g^{-1}=g_0\}$ is a subgroup of $G$. This is called the \emph{centralizer} of $g_0$ in $G$.

\item If $(a)$, $(b)$ are two given subgroups of $\Z$, we will define
\[
(a)+(b):=\{n\in \Z|n=x+y,~x\in (a),~y\in (b) \}.
\]
Show that 
\begin{itemize}
\item[(i)] $(a)+(b)$ is a subgroup of $\Z$.
\item[(ii)] Explicitly list the first 10 positive elements of the group $(4)+(6)$.
\item[(iii)] By a Theorem we proved in class, $(a)+(b)=(d)$ for some non-negative integer $d$. Show that $d=gcd(a,b)$. (Hint: Use the fact that, if $d=gcd(a,b)$, then, by the Euclidean algorithm, one can find integers $r,s\in \Z$ such that $d=ra+sb$.)
\end{itemize}

\item In class, we have define the group $\Z/(n)$ with the group addition given by
\[
\overline{a}+\overline{b}:=\overline{a+b}.
\] 
Show that this addition is well-defined (i.e.~it is independent of choices representative for the cosets). How many elements are there in the group? Justify your answer. 

\item Let $f:(\mathbb{R},+)\to (\mathbb{C}^*,\times)$ be the map $f(a)=e^{2\pi i a}$. Prove that it's a group homomorphism and determine its kernel and image.

\item Consider the group homomorphism $\mathrm{det}_n: GL_n(\R)\lra (\R\backslash \{0\},\times)$. Find its kernel and image.
\end{enumerate}
\end{document} 