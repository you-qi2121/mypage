\documentclass[12pt]{article}
\title{}
\date{}
\usepackage{amsmath,amsfonts,amsthm,amssymb,mathrsfs}
\usepackage{times}
%\usepackage{hyperref}
%\usepackage{CJKutf8}
\textwidth=16cm
\textheight=23cm
\voffset=-1.8cm
\hoffset=-1.1cm
%------------------------------------------------------------------------------%
\theoremstyle{plain}
\newtheorem{thm}{Theorem}[section]
\newtheorem{lem}[thm]{Lemma}
\newtheorem{cor}[thm]{Corollary}
\theoremstyle{definition}
\newtheorem{defn}{Definition}[section]
\theoremstyle{remark}
\newtheorem*{remark}{Remark}
\newtheorem*{Ex}{Example}
\newtheorem*{ex}{Exercise}

\newcommand{\ba}{\beta}
\newcommand{\da}{\delta}
\newcommand{\Da}{\Delta}
\newcommand{\ta}{\theta}
\newcommand{\Ta}{\Theta}
\newcommand{\la}{\lambda}
\newcommand{\Oa}{\Omega}
\newcommand{\vphi}{\varphi}
\newcommand{\N}{\mathbb{N}}
\newcommand{\Z}{\mathbb{Z}}
\newcommand{\R}{\mathbb{R}}
\newcommand{\C}{\mathbb{C}}
\newcommand{\lra}{\longrightarrow}

\newcommand{\pl}{\partial}
\newcommand{\fc}{\frac}
\newcommand{\na}{\nabla}
\newcommand{\ol}{\overline}
\newcommand{\lt}{\left}
\newcommand{\rt}{\right}
\newcommand{\rw}{\rightarrow}

\newcommand{\mf}{\mathbf}
\newcommand{\mb}{\mathbb}
\newcommand{\ml}{\mathcal}

\newcommand{\bs}{\boldsymbol}
\newcommand{\wt}{\widetilde}

\pagestyle{empty}

\begin{document}
\begin{center}
{\Large \bf Homework 3}
\end{center}
The work handed in should be entirely your own. You can consult any abstract algebra textbook (e.g.~Dummit and Foote, Artin), the course textbook and/or the class notes but nothing else. To receive full credit, justify your answer in a clear and logical way. Due Mar. 2.

\begin{enumerate}
\item Let $A$ and $B$ be matrices that are orthogonally conjugate to each other, i.e., there is a $g\in O_n(\R)$ such that $A=gBg^{-1}$.
\begin{enumerate}
\item[(a)] Show that $\mathrm{Tr}(A^tA)=\mathrm{Tr}(B^tB)$.
\item[(b)] Show that $\sum_{i,j=1}^n A_{ij}^2=\sum_{i,j=1}^n B_{ij}^2$.
\end{enumerate}

\item Let $S=\mathrm{M}_{m\times n}(\R)$ be the set of $m\times n$-matrices with coefficients in $\R$, and consider the group 
$$
G=GL_m(\R)\times GL_n(\R)=\{(g,h)|g\in GL_n(\R),h\in GL_m(\R)\}
$$ the cross product of the groups. Consider
\[
*:G\times S\lra S\quad, (g,h)*s:= gsh^{-1} .
\]
\begin{enumerate}
\item[(a)] Show that $*$ defines a group action.
\item[(b)] Describe the decomposition of $S$ into $G$ orbits.
\item[(c)] Suppose $m\leq n$. What is the stablizer group of the matrix $[I_m|0]$?
\end{enumerate}

\item Let $G$ be an abelian group. In class we defined the left, right and adjoint action of $G$ on itself. How many distinct orbits are there for each action?

\item The group of orthogonal transformations $O_n(\R)$ acts on the set of lines in $\R^n$. 
\begin{enumerate}
\item Write this action in terms of the formal definition of group actions on a set we defined in class.
\item Let $l=\{(a,0,\dots,0)|a\in \R\}$ be the line of the first coordinate axis. Find the stablizer group of $l$.
\end{enumerate}

\item Show that $H_2^n$ has $C_2^n$ as a normal subgroup inside it.

\item
\begin{enumerate}
\item[(a)] Prove that, given any element of a group $g\in G$, there is a unique group homomorphism $\phi_g$ determined by $g$:
\[
\phi_g: \Z\lra G,\quad n\mapsto g^n.
\]
\item[(b)] Consider $G=SO(2,\mathbb{R})$, and pick 
    $$g=
    \left(
    \begin{array}{cc}
    \cos \theta &-\sin \theta\\
    \sin \theta & \cos \theta
    \end{array}
    \right)
    $$
   Here $\theta$ is a fixed angle. Write down explicitly the group homomorphism determined by this element. When is this homomorphism injective?
\end{enumerate}

\item In this exercise, you will show one important ingredient in the course of our classification of finite subgroups of $SO_3(\R)$. Recall that a polyhedron in $\R^3$ is called \emph{regular} if for the same number of edges emanate from each vertex, and each facet of the polyhedron has the same number of edges. 

Use Euler's formula 
\[
V-E+F=2,
\]
where $V$ is the number of vertices of (any) polyhedron, $E$ is the number of edges and $F$ is the number of facets, to show that there are only five regular polyhedra in $\R^3$.
\end{enumerate}
\end{document} 