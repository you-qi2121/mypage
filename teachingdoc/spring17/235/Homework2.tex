\documentclass[12pt]{article}
\title{}
\date{}
\usepackage{amsmath,amsfonts,amsthm,amssymb,mathrsfs}
\usepackage{times}
%\usepackage{hyperref}
%\usepackage{CJKutf8}
\textwidth=16cm
\textheight=23cm
\voffset=-1.8cm
\hoffset=-1.1cm
%------------------------------------------------------------------------------%
\theoremstyle{plain}
\newtheorem{thm}{Theorem}[section]
\newtheorem{lem}[thm]{Lemma}
\newtheorem{cor}[thm]{Corollary}
\theoremstyle{definition}
\newtheorem{defn}{Definition}[section]
\theoremstyle{remark}
\newtheorem*{remark}{Remark}
\newtheorem*{Ex}{Example}
\newtheorem*{ex}{Exercise}

\newcommand{\ba}{\beta}
\newcommand{\da}{\delta}
\newcommand{\Da}{\Delta}
\newcommand{\ta}{\theta}
\newcommand{\Ta}{\Theta}
\newcommand{\la}{\lambda}
\newcommand{\Oa}{\Omega}
\newcommand{\vphi}{\varphi}
\newcommand{\N}{\mathbb{N}}
\newcommand{\Z}{\mathbb{Z}}
\newcommand{\R}{\mathbb{R}}
\newcommand{\C}{\mathbb{C}}
\newcommand{\lra}{\longrightarrow}

\newcommand{\pl}{\partial}
\newcommand{\fc}{\frac}
\newcommand{\na}{\nabla}
\newcommand{\ol}{\overline}
\newcommand{\lt}{\left}
\newcommand{\rt}{\right}
\newcommand{\rw}{\rightarrow}

\newcommand{\mf}{\mathbf}
\newcommand{\mb}{\mathbb}
\newcommand{\ml}{\mathcal}

\newcommand{\bs}{\boldsymbol}
\newcommand{\wt}{\widetilde}

\pagestyle{empty}

\begin{document}
\begin{center}
{\Large \bf Homework 2}
\end{center}
The work handed in should be entirely your own. You can consult any abstract algebra textbook (e.g.~Dummit and Foote, Artin), the course textbook and/or the class notes but nothing else. To receive full credit, justify your answer in a clear and logical way. Due Feb. 16.

\begin{enumerate}

\item Construct an injective group homomorphism from the cyclic group $C_4$ to the symmetric group $S_4$. 
Describe its image in $S_4$ in  terms of the cycle notation. How many different injective homomorphisms 
from $C_4$ to $S_4$ can you define? 

\item Let $V$ be a Euclidean vector space, and let $\mathrm{Iso}(V):=\{\phi|\phi~\textrm{is an isometry of $V$}\}$. 
\begin{enumerate}
\item Prove that $\mathrm{Iso}(V)$ forms a group under composition of maps.
\item Let $\mathrm{Tran}(V)\subset \mathrm{Iso}(V)$ be the subset of maps that are translations. Recall that a translation on $V$ is a map $t_{\mathbf{v}_0}: V\lra V,~\mathbf{v}\mapsto \mathbf{v}+\mathbf{v}_0$ for some fixed vector $\mathbf{v}_0$ determined by $t_{\mathbf{v}_0}$. Show that $\mathrm{Tran}(V)$ is a normal subgroup of $\mathrm{Iso}(V)$.
\end{enumerate}

\item Prove the following claim we have made in class. 

Let $V$ be a Euclidean vector space with an inner product $\cdot$ (to differentiate with the standard inner product on $\R^n$, let's use a different notation here):
\[
\cdot: V\times V\lra \R, \quad \mathbf{u}, \mathbf{v}\mapsto \mathbf{u}\cdot \mathbf{v}.
\]
Suppose $\beta=\{\mathbf{v}_1,\dots, \mathbf{v}_n\}$ is an orthonormal basis for $V$, i.e., it satisfies
\[
\mathbf{v}_i\cdot \mathbf{v}_j=\delta_{i,j}.
\]
Prove that the parametrization isomorphism
\[
\Psi_{\beta}: V\lra \R^n, \quad \mathbf{v}=\sum_{i=1}^n a_i\mathbf{v}_i\mapsto (a_1,\dots, a_n)^t
\]
is an \emph{isometry} in the sense that, for any $\mathbf{u}, \mathbf{v} \in V$, we have
\[
\mathbf{u}\cdot \mathbf{v} = (\Psi_\beta(\mathbf{u}), \Psi_\beta(\mathbf{v}))_{\R^n},
\] 
where the right hand side stands for the standard inner product for vectors in $\R^n$.


\item Recall that to prove a statement that involves a natural number $n$, the method of induction can be used: 
\begin{enumerate} 
\item Show that the statement holds for $n =1$. 
\item Assuming that the statement holds for a given natural $n$, show that it also holds for $n+1$. 
\end{enumerate} 
From here it follows that the statement holds for all natural $n$.  


Prove by induction that for all natural numbers $n$, and a given angle $0 \leq \vartheta < 2 \pi$, the following matrix equality holds: 
\[ \left( \begin{array}{cc}  
        \cos{\vartheta}  &  - \sin{\vartheta} \\ 
        \sin{\vartheta}   &    \cos{\vartheta}   \end{array} \right)^n  = 
        \left( \begin{array}{cc} 
        \cos{n \vartheta} & - \sin{n \vartheta} \\ 
        \sin{n \vartheta}  &   \cos{n \vartheta}  \end{array}  \right)  . \] 
Give a geometric interpretation of this identity. 

\item Diagonalize the matrix
\[ 
S_1 = 
\left( 
\begin{array}{cc}  
-\frac{1}{2} & \frac{\sqrt{3}}{2} \\ 
\frac{\sqrt{3}}{2} & \frac{1}{2}  
\end{array} 
\right)  
\] 


\item Consider the reflections in $\R^2$ given by the matrices 
\[ \begin{array}{cc} S_0 = \left( \begin{array}{cc} 
               1  & 0  \\ 
              0   & -1 \end{array} 
            \right)       &  
            S_1 = \left( \begin{array}{cc}  
            -\frac{1}{2} & \frac{\sqrt{3}}{2} \\ 
            \frac{\sqrt{3}}{2} & \frac{1}{2}  \end{array} 
            \right)  \\  \end{array}  \] 
Check that  $S_0$ and $S_1$ are reflections and find the reflection axes. What group do they generate? 
Find all group elements and write down a multiplication table between them. 

{\it Hint:} Since $S_0^2 = S_1^2 = 1$,  the only nontrivial elements 
of the group generated by $S_0$ and $S_1$ are the products where $S_0$ and $S_1$ alternate.  
Find all distinct elements of this form and determine their products. 

\item In this exercise, we will fill out the details of a Lemma we claimed in class.

Consider $\R^n$ with the standard Euclidean inner product. An operator $T:\R^n \lra \R^n$ is called \emph{normal} if $T^tT=TT^t$. For instance, if $T$ is orthogonal, it is normal.
\begin{enumerate}
\item[(i)] If $\mathbf{u}$ is an eigenvector for $T$ with eigenvalue $\lambda\in \R$, i.e.,
\[T(\mathbf{u})=\lambda \mathbf{u}\]
show that $\mathbf{u}$ is also an eigenvector for $T^t$ with the same eigenvalue. (Hint: Prove that $||T^t(\mathbf{u})-\lambda \mathbf{u} \||=0$.)
\item[(ii)] Show that if $\mathbf{u}$ and $\mathbf{v}$ are eigenvectors of $T$ with distinct eigenvalues $\lambda$ and $\mu$ respectively, then $\mathbf{u}\perp \mathbf{v}$.
\end{enumerate} 

\end{enumerate}
\end{document} 