\documentclass[12pt]{article}
\title{}
\date{}
\usepackage{amsmath,amsfonts,amsthm,amssymb,mathrsfs}
\usepackage{times}
\usepackage{diagcat}
%\usepackage{hyperref}
%\usepackage{CJKutf8}
\textwidth=16cm
\textheight=23cm
\voffset=-1.8cm
\hoffset=-1.1cm
%------------------------------------------------------------------------------%
\theoremstyle{plain}
\newtheorem{thm}{Theorem}[section]
\newtheorem{lem}[thm]{Lemma}
\newtheorem{prop}[thm]{Proposition}
\newtheorem{cor}[thm]{Corollary}
\theoremstyle{definition}
\newtheorem{defn}{Definition}[section]
\theoremstyle{remark}
\newtheorem*{remark}{Remark}
\newtheorem*{Ex}{Example}
\newtheorem*{ex}{Exercise}

\newcommand{\ba}{\beta}
\newcommand{\da}{\delta}
\newcommand{\Da}{\Delta}
\newcommand{\ta}{\theta}
\newcommand{\Ta}{\Theta}
\newcommand{\la}{\lambda}
\newcommand{\Oa}{\Omega}
\newcommand{\vphi}{\varphi}
\newcommand{\N}{\mathbb{N}}
\newcommand{\Z}{\mathbb{Z}}
\newcommand{\R}{\mathbb{R}}
\newcommand{\C}{\mathbb{C}}
\newcommand{\lra}{\longrightarrow}

\newcommand{\pl}{\partial}
\newcommand{\fc}{\frac}
\newcommand{\na}{\nabla}
\newcommand{\ol}{\overline}
\newcommand{\lt}{\left}
\newcommand{\rt}{\right}
\newcommand{\rw}{\rightarrow}

\newcommand{\mf}{\mathbf}
\newcommand{\mb}{\mathbb}
\newcommand{\ml}{\mathcal}

\newcommand{\bs}{\boldsymbol}
\newcommand{\wt}{\widetilde}

\pagestyle{empty}

\begin{document}
\begin{center}
{\Large \bf Homework 6}
\end{center}
The work handed in should be entirely your own. You can consult any abstract algebra textbook (e.g.~Dummit and Foote, Artin), the course textbook and/or the class notes but nothing else. To receive full credit, justify your answer in a clear and logical way. Due April.~25.

\begin{enumerate}
\item Read the textbook Chapter 5 on classification of Coxeter groups.
\item Show that the reflection about a root $\mathbf{r}$
\[
S_{\mathbf{r}}(\mathbf{v}):=\mathbf{v}-2\frac{(\mathbf{v},\mathbf{r})}{||\mathbf{r}||}\mathbf{r}
\]
is an orthogonal transformation.

\item Suppose $\Pi$ is a base for $\Delta$.
\begin{enumerate}
\item[(a)] If $\mathbf{r}\in \Delta^+$, show that $\mathbf{r}\in \Pi$ if and only if $\mathbf{r}$ is not a strictly positive linear combination of two or more positive roots.
\item[(b)] Use part (a) to give an alternative proof of the uniqueness of $\Pi$.
\end{enumerate}

\item Show that the vector $\rho$ we defined in class
\[
\rho:=\frac{1}{2}\sum_{\mathbf{r}\in \Delta^+} \mathbf{r}
\]
satisfies $(\rho,\mathbf{r})>0$ for all $\mathbf{r}\in \Delta^+$.

\item Let $H_2^m$ be the dihedral reflection group on $\R^2$. Find a set of simple roots for $H_2^m$ and compute its Coxeter graph. Compute the determinant of the Coxeter matrix.

\item This is a continuation of the exercise from last week about $S_4$. Use what you have proven there to compute the Coxeter graph for $S_4$. Compute the determinant of the Coxeter matrix. 

\item Verify the determinant formula for $I_4$.




\end{enumerate}
\end{document} 