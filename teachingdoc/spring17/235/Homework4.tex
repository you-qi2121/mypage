\documentclass[12pt]{article}
\title{}
\date{}
\usepackage{amsmath,amsfonts,amsthm,amssymb,mathrsfs}
\usepackage{times}
%\usepackage{hyperref}
%\usepackage{CJKutf8}
\textwidth=16cm
\textheight=23cm
\voffset=-1.8cm
\hoffset=-1.1cm
%------------------------------------------------------------------------------%
\theoremstyle{plain}
\newtheorem{thm}{Theorem}[section]
\newtheorem{lem}[thm]{Lemma}
\newtheorem{prop}[thm]{Proposition}
\newtheorem{cor}[thm]{Corollary}
\theoremstyle{definition}
\newtheorem{defn}{Definition}[section]
\theoremstyle{remark}
\newtheorem*{remark}{Remark}
\newtheorem*{Ex}{Example}
\newtheorem*{ex}{Exercise}

\newcommand{\ba}{\beta}
\newcommand{\da}{\delta}
\newcommand{\Da}{\Delta}
\newcommand{\ta}{\theta}
\newcommand{\Ta}{\Theta}
\newcommand{\la}{\lambda}
\newcommand{\Oa}{\Omega}
\newcommand{\vphi}{\varphi}
\newcommand{\N}{\mathbb{N}}
\newcommand{\Z}{\mathbb{Z}}
\newcommand{\R}{\mathbb{R}}
\newcommand{\C}{\mathbb{C}}
\newcommand{\lra}{\longrightarrow}

\newcommand{\pl}{\partial}
\newcommand{\fc}{\frac}
\newcommand{\na}{\nabla}
\newcommand{\ol}{\overline}
\newcommand{\lt}{\left}
\newcommand{\rt}{\right}
\newcommand{\rw}{\rightarrow}

\newcommand{\mf}{\mathbf}
\newcommand{\mb}{\mathbb}
\newcommand{\ml}{\mathcal}

\newcommand{\bs}{\boldsymbol}
\newcommand{\wt}{\widetilde}

\pagestyle{empty}

\begin{document}
\begin{center}
{\Large \bf Homework 4}
\end{center}
The work handed in should be entirely your own. You can consult any abstract algebra textbook (e.g.~Dummit and Foote, Artin), the course textbook and/or the class notes but nothing else. To receive full credit, justify your answer in a clear and logical way. Due Mar.~30.

\begin{enumerate}
\item The most important exercise is that you should find some time during the break to review what we have learnt so far. Also, read Chapter 4 on fundamental regions.

\item Prove the following Proposition we stated in class.
\begin{prop}
Let $G$ be a group acting on a set $S$, and let $s,t\in S$ be two elements lying in the same orbit, i.e., there is a $g\in G$ such that $g*s=t$. Then there is an isomorphism of stablizer subgroups
\[
\Phi_g: Z_G(s)\lra Z_G(t),\quad x\mapsto gxg^{-1}.
\]
\end{prop}
In particular, if the stablizer groups are finite, then $|Z_G(s)|=|Z_G(t)|$.

\item Use the stablizer-orbit counting formula $|G|=|Z_G(s)||O_{s}|$ for a finite transitive group action to give an alternative count of the number of elements in the dihedral group $D_2^n$. (Hint: Consider the action of the dihedral group on the set of sides or vertices of a regular $n$gon and use the formula)
.
\item Use the following inductive procedure to prove that $|S_n|=n!$.
\begin{enumerate}
\item[(1)] Show that the group $S_n$ acts transitively on the set $I_n=\{1,\dots, n\}$, and determine the stablizer subgroup of the element $n$.
\item[(2)] Use part (1) and induction ($S_1$ obviously has only one element) to prove the desired formula.
\end{enumerate}

\item Show that the product of two reflections of $\R^2$ is a rotation through
twice the angle between their reflecting lines. More precisely,
say that $S_i$ has reflecting line at angle $\theta_i$ with the positive $x$-axis,
with $0 \leq \theta_i< \pi$, and say that $\theta_1<\theta_2$, Then $S_2S_1$ is a counterclockwise
rotation through angle $2(\theta_1-\theta_2)$, and $S_1S_2$ is a clockwise
rotation through angle $2(\theta_2 - \theta_1)$.

\end{enumerate}
\end{document} 