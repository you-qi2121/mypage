\documentclass[12pt]{article}
\title{}
\date{}
\usepackage{amsmath,amsfonts,amsthm,amssymb,mathrsfs}
\usepackage{times}
\usepackage{diagcat}
%\usepackage{hyperref}
%\usepackage{CJKutf8}
\textwidth=16cm
\textheight=23cm
\voffset=-1.8cm
\hoffset=-1.1cm
%------------------------------------------------------------------------------%
\theoremstyle{plain}
\newtheorem{thm}{Theorem}[section]
\newtheorem{lem}[thm]{Lemma}
\newtheorem{prop}[thm]{Proposition}
\newtheorem{cor}[thm]{Corollary}
\theoremstyle{definition}
\newtheorem{defn}{Definition}[section]
\theoremstyle{remark}
\newtheorem*{remark}{Remark}
\newtheorem*{Ex}{Example}
\newtheorem*{ex}{Exercise}

\newcommand{\ba}{\beta}
\newcommand{\da}{\delta}
\newcommand{\Da}{\Delta}
\newcommand{\ta}{\theta}
\newcommand{\Ta}{\Theta}
\newcommand{\la}{\lambda}
\newcommand{\Oa}{\Omega}
\newcommand{\vphi}{\varphi}
\newcommand{\N}{\mathbb{N}}
\newcommand{\Z}{\mathbb{Z}}
\newcommand{\R}{\mathbb{R}}
\newcommand{\C}{\mathbb{C}}
\newcommand{\lra}{\longrightarrow}

\newcommand{\pl}{\partial}
\newcommand{\fc}{\frac}
\newcommand{\na}{\nabla}
\newcommand{\ol}{\overline}
\newcommand{\lt}{\left}
\newcommand{\rt}{\right}
\newcommand{\rw}{\rightarrow}

\newcommand{\mf}{\mathbf}
\newcommand{\mb}{\mathbb}
\newcommand{\ml}{\mathcal}

\newcommand{\bs}{\boldsymbol}
\newcommand{\wt}{\widetilde}

\pagestyle{empty}

\begin{document}
\begin{center}
{\Large \bf Homework 5}
\end{center}
The work handed in should be entirely your own. You can consult any abstract algebra textbook (e.g.~Dummit and Foote, Artin), the course textbook and/or the class notes but nothing else. To receive full credit, justify your answer in a clear and logical way. Due April.~6.

\begin{enumerate}
\item Read the textbook Chapter 4 on root systems.

\item Prove the statement we made in class: Let $P\subset V$ be a hyperplane orthogonal to the vector $\mathbf{r}\in V$:
\[
P_\mathbf{r}=\{\mathbf{v}\in V|(\mathbf{v},\mathbf{r})=0\}.
\] 
Let $g:V\lra V$ be an orthogonal transformation. Show that the hyperplane orthogonal to $g(\mathbf{r})$ coincides with
\[
P_{g(\mathbf{r})}=g(P_\mathbf{r}):=\{g(\mathbf{v})|\mathbf{v}\in P_\mathbf{r}\}.
\]

\item Let $V_1$ and $V_2$ be two subspaces of a Euclidean vector space $V$. Prove that
\begin{itemize}
\item $(V_1^{\perp})^\perp=V_1$,
\item $ (V_1+V_2)^{\perp}=V_1^\perp \cap V_2^\perp$,
\item $(V_1\cap V_2)^\perp = V_1^\perp+V_2^\perp$.
\end{itemize}

\item In this exercise, you will be asked to check the reflection group structure of $S_4$, the symmetric group on four letters.
\begin{enumerate}
\item[(1)] Recall that $S_4$ acts on the set $\{1,2,3,4\}$ by permuting the letters. We ``linearize'' the action by constructing a (Euclidean) vector space $\R^4$ with the standard coordinate bases $\{e_1,e_2, e_3, e_4\}$, and let $S_4$ permute the indices of the basis vector. Show that in this way, $S_4$ is a subgroup of $O_4(\R)$.
\item[(2)] Using the diagrammatic notation we have developed in class, justify that any element of $S_4$ can be generated by the following three elements
\[
\sigma_1:=
\begin{DGCpicture}
\DGCstrand(0,0)(0.5,1)
\DGCstrand(0.5,0)(0,1)
\DGCstrand(1,0)(1,1)
\DGCstrand(1.5,0)(1.5,1)
\end{DGCpicture},\quad \quad
\sigma_2:=
\begin{DGCpicture}
\DGCstrand(0,0)(0,1)
\DGCstrand(0.5,0)(1,1)
\DGCstrand(1,0)(0.5,1)
\DGCstrand(1.5,0)(1.5,1)
\end{DGCpicture},\quad\quad
\sigma_3:=
\begin{DGCpicture}
\DGCstrand(0,0)(0,1)
\DGCstrand(0.5,0)(0.5,1)
\DGCstrand(1,0)(1.5,1)
\DGCstrand(1.5,0)(1,1)
\end{DGCpicture}.
\]
\item[(3)] How many reflection elements are there in $S_4$? Can you find an order-two element that is not a reflection?
\item[(4)] Find roots for $\sigma_i$ ($i=1,2,3$).
\item[(5)] Find all roots for $S_4$. (Hint: Use the group action on the roots you have found in part (4)).
\item[(6)] Is this action of $S_4$ on $\R^4$ effective? What is the point-wise fixed space 
\[
V_0=\{\mathbf{v}\in \R^4|s(\mathbf{v})=\mathbf{v},~\forall s\in S_4\}?
\]
\end{enumerate}


\end{enumerate}
\end{document} 