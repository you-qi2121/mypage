\documentclass[margin,line]{res}
\usepackage{url}

\oddsidemargin -.5in \evensidemargin -.5in \textwidth=6.0in
\itemsep=0in
\parsep=0in

\newenvironment{list1}{
  \begin{list}{\ding{113}}{%
      \setlength{\itemsep}{0in}
      \setlength{\parsep}{0in} \setlength{\parskip}{0in}
      \setlength{\topsep}{0in} \setlength{\partopsep}{0in}
      \setlength{\leftmargin}{0.17in}}}{\end{list}}
\newenvironment{list2}{
  \begin{list}{$\bullet$}{%
      \setlength{\itemsep}{0in}
      \setlength{\parsep}{0in} \setlength{\parskip}{0in}
      \setlength{\topsep}{0in} \setlength{\partopsep}{0in}
      \setlength{\leftmargin}{0.2in}}}{\end{list}}


\begin{document}

\name{Syllabus of Modern Algebra II \vspace*{.1in}}

\begin{resume}

\section{\sc Key Information}
\vspace{.05in}
\begin{tabular}{@{}p{3.4in}p{4in}}
 Instructor: You Qi            & Tel: (203) 432-6859   \\
Time: Every Tue, Thu 1:00--2:15 pm & Venue: 431 DL, 10 Hillhouse\\
Office: 404 DL, 10 Hillhouse   & E-mail:  you.qi@yale.edu \\
Office Hour: Tue, Thu 10:30--11:30 am &\\
& \\
%Teaching Assistant: Efim Abrikosov & E-mail: efim.abrikosov@yale.edu\\
%TA Office Hour: Wed, 9:00--10:30am  & Venue: 206 LOM\\
\end{tabular}
Course webpage: \url{http://math.yale.edu/~yq64/spring2016381}.

\section{\sc Course Description}
This course is aimed at giving a gentle introduction to noncommutative algebra. We aim to illustrate basic principles of abstract algebra via representation theory of some finite groups and finite-dimensional Lie algebras.

\section{\sc Textbooks}
\begin{itemize}
\item[(1)] J. P. Serre, \emph{Linear Representations of Finite Groups}, GTM 42, Springer.

\item[(2)] W. Fulton and J. Harris, \emph{Representation Theory, A First Course}, GTM 129, Springer. 

\item[(3)] J. Humphreys, \emph{Introduction to Lie Algebras and Representation Theory}, GTM 9, Springer.
\end{itemize}

\section{\sc Prerequisites}
Basic undergraduate level abstract algebra, commutative algebra as in the first part of the course, some famliarity with finite groups, character theory will be assumed (see the first two chapters of (1)).


\section{\sc Grading}
There will be a few homeworks assigned every several weeks when we finish a topic.
\begin{list2}
\item Homeworks, 10 \%
\item Midterm, 40 \%
\item Final, 50 \%
\end{list2}

\end{resume}

\end{document}



