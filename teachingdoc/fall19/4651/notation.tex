\documentclass[11pt]{article}
\usepackage[all,2cell]{xy} \UseAllTwocells \SilentMatrices
\usepackage{latexsym,amsfonts,amssymb}
\usepackage{amsmath,amsthm,amscd}
\usepackage{hyperref,psfrag}
%\usepackage{diagcat}
\usepackage{color}
\usepackage{etoolbox}

\usepackage[dvips]{epsfig}

\usepackage{psfrag}
\def\drawing#1{\begin{center}\epsfig{file=#1}\end{center}}
% including eps files

\usepackage{graphicx}

\usepackage{a4wide}
\renewcommand*\rmdefault{ppl}\normalfont\upshape

%\usepackage{fancyhdr}
%\pagestyle{fancyplain}
%\renewcommand{\sectionmark}[1]{\markboth{#1}{}}
%\renewcommand{\subsectionmark}[1]{\markright{#1}}
%\lhead[\fancyplain{}{\bfseries\thepage}]{\fancyplain{}{\sl\bfseries\rightmark}}
%\rhead[\fancyplain{}{\sl\bfseries\leftmark}]{\fancyplain{}{\bfseries\thepage}}
%\cfoot{}

\hfuzz=6pc


%------------------------------------------------------------------------------%
\theoremstyle{plain}
\newtheorem{thm}{Theorem}
\newtheorem{prop}[thm]{Proposition}
\newtheorem{lemma}[thm]{Lemma}
\newtheorem{cor}[thm]{Corollary}
\newtheorem{conj}[thm]{Conjecture}

\theoremstyle{definition}

\newtheorem{defn}[thm]{Definition}
\newtheorem{eg}[thm]{Example}
\newtheorem{rmk}[thm]{Remark}
\newtheorem{ntn}[thm]{Notation}
\newtheorem{ex}[thm]{Exercise}

\usepackage{bbm}
\def\1{\mathbbm 1}
\def\R{\mathbb R}
\def\Q{\mathbb Q}
\def\Z{\mathbb Z}
\def\N{\mathbb N}
\def\C{\mathbb C}
\def\F{\mathbb F}
\def\P{\mathbb P}
\def\o{\otimes}
\def\lra{\longrightarrow}
\def\zb{\overline{z}}
\def\Arg{\mathrm{Arg}}
\def\Log{\mathrm{Log}}

\begin{document}
\begin{center}
{\Large \bf A Summary of Commonly Used Math Notations}
\end{center}

\begin{itemize}
\item $\in$: This indicates an element belonging to a set. Example: ``$5$ is a natural number'' can be written as ``$5\in \N$''. Compare with the subset notation below.
\item $\subset$: This indicates the containment relationship between sets. For instance, ``$\{1,2\}$ is a subset of $\{1,2,3,4\}$'' can be abbreviated as $\{1,2\}\subset \{1,2,3,4\}$. There are some variations of this notation, like $\supset$ (containment), $\subseteq$ (subset, may be equal), $\supseteq$ (containment, may be equal), $\subsetneq$ (subset and not equal) $\supsetneq$. Notice that this is a relationship between sets, while the notation $\in$ is between an element and a set. For instance, 
\[
\{5\} \subset \{1,2,3,4,5\}, \quad \quad 5\in \{1,2,3,4,5\}
\]
both tell you that the element 5 is in the set $\{1,2,3,4,5\}$. But $\{5\}\in \{1,2,3,4,5\}$ is NOT mathematically correct.
\item $\forall$: This means ``for any'' or ``for all.'' Example: ``For any vector $v$ in a vector space $V$, a scalar multiple of it is still in the vector space" can be written as ``$\forall v\in V$, and $\forall c\in \F$, $cv\in \F$.''
\item $\exists$: This means ``there exists." For example: ``For any $\epsilon>0$, there exists a $\delta>0$ such that..." can be written as ``$\forall \epsilon>0,~\exists \delta>0$ s.t.~...'' A negation of this symbol is $\nexists$, meaning "there does not exist."
\item $\Rightarrow$ means the statement before the arrow implies the statement after the arrow. $\Leftrightarrow$ indicates the equivalence of statements.
\item $\N$: the set of natural numbers $\N=\{0,1,2,3,4,\dots\}$. $\Z$: the set of integers. $\Q$: the set of rational numbers (this is the first example of a field). $\R$: the set of real numbers. $\C$: the set of complex numbers.
\item $\sum$ and $\prod$: meaning taking sum/product of all terms behind the symbol satisfying some conditions. For instance, summing over all natural numbers from $0$ to $100$ can be written as
    \[
    0+1+\dots+100=\sum_{k=0}^{100} k.
    \]
\item Greek letters : $\alpha,\beta,\gamma,\delta,\epsilon,\zeta,\eta,\theta,\kappa,\lambda$ etc. Used as alternatives for English letters. In math different alphabets are usually used to represent concepts of different nature.
\item $\cup$: union of sets $A\cup B=\{x|x\in A~\textrm{or}~x\in B\}$. 
\item $\cap$: intersection of sets $A\cap B=\{x|x\in A~\textrm{and}~x\in B\}$.
\end{itemize}

\end{document} 