\documentclass[11pt]{article}
\usepackage[all,2cell]{xy} \UseAllTwocells \SilentMatrices
\usepackage{latexsym,amsfonts,amssymb}
\usepackage{amsmath,amsthm,amscd}
\usepackage{hyperref,psfrag}
%\usepackage{diagcat}
\usepackage{color}
\usepackage{etoolbox}

\usepackage[dvips]{epsfig}

\usepackage{psfrag}
\def\drawing#1{\begin{center}\epsfig{file=#1}\end{center}}
% including eps files

\usepackage{graphicx}

\usepackage{a4wide}
\renewcommand*\rmdefault{ppl}\normalfont\upshape

%\usepackage{fancyhdr}
%\pagestyle{fancyplain}
%\renewcommand{\sectionmark}[1]{\markboth{#1}{}}
%\renewcommand{\subsectionmark}[1]{\markright{#1}}
%\lhead[\fancyplain{}{\bfseries\thepage}]{\fancyplain{}{\sl\bfseries\rightmark}}
%\rhead[\fancyplain{}{\sl\bfseries\leftmark}]{\fancyplain{}{\bfseries\thepage}}
%\cfoot{}

\hfuzz=6pc


%------------------------------------------------------------------------------%
\theoremstyle{plain}
\newtheorem{thm}{Theorem}
\newtheorem{prop}[thm]{Proposition}
\newtheorem{lemma}[thm]{Lemma}
\newtheorem{cor}[thm]{Corollary}
\newtheorem{conj}[thm]{Conjecture}

\theoremstyle{definition}

\newtheorem{defn}[thm]{Definition}
\newtheorem{eg}[thm]{Example}
\newtheorem{rmk}[thm]{Remark}
\newtheorem{ntn}[thm]{Notation}
\newtheorem{ex}[thm]{Exercise}

\usepackage{bbm}
\def\1{\mathbbm 1}
\def\R{\mathbb R}
\def\Q{\mathbb Q}
\def\Z{\mathbb Z}
\def\N{\mathbb N}
\def\C{\mathbb C}
\def\F{\mathbb F}
\def\P{\mathbb P}
\def\o{\otimes}
\def\lra{\longrightarrow}
\def\zb{\overline{z}}
\def\Arg{\mathrm{Arg}}
\def\Log{\mathrm{Log}}

\begin{document}
\begin{center}
{\Large \bf Exercises for Week 6}
\end{center}
The work handed in should be entirely your own. You can consult the textbook and/or the class notes but nothing else. To receive full credit, justify your answer in a clear and logical way. Due Oct.~9.

\paragraph{Reading.} Read Sections 2.5, 3.1, 3.2 of the textbook carefully.

\begin{enumerate}
%\item Section 3.1 Exercise 1, 2, 5
%\item Section 3.2 Exercise 1, 2 (a), (d), (e)
\item[(1)] This exercise is intended to develop a useful notation to help you understand/memorize change of basis matrices.

Let $T:V\lra W$ be a linear map, and let $\beta=\{v_1,\dots, v_n\}$, $\beta^\prime=\{v_1^\prime,\dots, v_n^\prime\}$ be ordered bases for $V$ and $\gamma=\{w_1,\dots, w_m\}$, $\gamma^\prime=\{w_1^\prime,\dots, w_m^\prime\}$ be ordered bases for $W$.

We know how to parametrize vectors in $V$ (and similarly for $W$) by
\[
v\in V \Longleftrightarrow v=\sum_{i=1}^n a_iv_i = \beta \cdot [v]^{\beta}= (v_1,\dots, v_n) \cdot
\left( 
\begin{matrix}
a_1\\
\vdots\\
a_n
\end{matrix}
\right).
\]
Likewise, in terms of $\beta^\prime$, we have
\[
 v=\sum_{i=1}^n a_iv_i = \beta^\prime \cdot [v]^{\beta^\prime}= (v_1^\prime,\dots, v_n^\prime) \cdot
\left( 
\begin{matrix}
a_1^\prime\\
\vdots\\
a_n^\prime
\end{matrix}
\right).
\]

(a) Show that the change of basis matrix $Q:= [\mathrm{Id_V}]_\beta^{\beta^\prime}=(q_{i,j})$ has the effect
\[
\beta = \beta^\prime\cdot Q \Longleftrightarrow
(v_1,\dots, v_n)= (v_1^\prime,\dots, v_n^\prime) \cdot
\left( 
\begin{matrix}
q_{1,1} & q_{1,2} &\dots & q_{1,n}\\
q_{2,1} & q_{2,2} &\dots & q_{2,n}\\
\vdots & \vdots &\ddots & \vdots\\
q_{n,1} & q_{n,2} &\dots & q_{n,n}
\end{matrix}
\right).
\]
Thus $\beta^\prime \cdot [v]^{\beta^\prime} = v = \beta \cdot [v]^\beta$ implies that
\[
\beta^\prime \cdot [v]^{\beta^\prime} = v = \beta \cdot Q [v]^\beta \Rightarrow   [v]^{\beta^\prime}=  Q [v]^\beta.
\]

(b) Define $T(\beta):= (T(v_1),T(v_2),\dots, T(v_n))$, a row of vectors. Show that
\[
T(\beta)= (w_1,\dots, w_m)\cdot A.
\]
where $A= [T]_\beta^\gamma $ is the matrix of $T$ with respect to $\beta$ and $\gamma$.

(c) Now, if $\beta = \beta^\prime \cdot Q_1$ and $\gamma = \gamma^\prime \cdot Q_2$ where 
$Q_1=[\mathrm{Id}_V]_\beta^{\beta^\prime}$ and $Q_2=[\mathrm{Id}_W]_\gamma^{\gamma^\prime}$ are the respective change of coordinate matrices, we then have
\[
T(\beta) = T(\beta^\prime \cdot Q_1)= T(\beta^\prime) \cdot Q_1
\]
since $T$ is linear. Combine this with part (b) and show that
\[
A^\prime = Q_2^{-1} A Q_1,
\]
where $A^\prime = [T]_{\beta^\prime}^{\gamma^\prime}$.
\item[(2)] 3.1 Exercises 1, 6.
\item[(3)] 3.2 Exercise 1, 3, 5 (b), (c), 7, 8.
\item[(4)] Prove that $\mathrm{rank}(A)=\mathrm{rank}(A^t)$ (Reduce both $A$ and $A^t$ into the simplest possible form via row and column operations. This is Corollary 2 of Section 2, but try not to look at the proof and directly prove it yourself.)
\end{enumerate}
\end{document} 