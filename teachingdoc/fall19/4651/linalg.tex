\documentclass[margin,line]{res}
\usepackage{url}

\oddsidemargin -.5in \evensidemargin -.5in \textwidth=6.0in
\itemsep=0in
\parsep=0in

\newenvironment{list1}{
  \begin{list}{\ding{113}}{%
      \setlength{\itemsep}{0in}
      \setlength{\parsep}{0in} \setlength{\parskip}{0in}
      \setlength{\topsep}{0in} \setlength{\partopsep}{0in}
      \setlength{\leftmargin}{0.17in}}}{\end{list}}
\newenvironment{list2}{
  \begin{list}{$\bullet$}{%
      \setlength{\itemsep}{0in}
      \setlength{\parsep}{0in} \setlength{\parskip}{0in}
      \setlength{\topsep}{0in} \setlength{\partopsep}{0in}
      \setlength{\leftmargin}{0.2in}}}{\end{list}}


\begin{document}

\name{Syllabus of Advanced Linear Algebra \vspace*{.1in}}

\begin{resume}

\section{\sc Basics}
\vspace{.05in}
\begin{tabular}{@{}p{3.4in}p{4in}}
 Instructor: You Qi            & Tel: (434) 924-4936   \\
Time: Every Mon, Wed 2:00--3:15 pm & Venue: New Cabell Hall 032\\
Office: Kerchoff Hall 305   & E-mail: yq2dw@virginia.edu \\
Office Hour: Mon 3:30--5:00 pm, Tue 10:00-11:30am &\\
\end{tabular}
Course webpage: \url{http://people.virginia.edu/~yq2dw/fall20194651.html}.

\section{\sc Important Dates}
\begin{tabular}{@{}p{3.4in}p{4in}}
Last day to drop class: September 10th. & \\
Last day to withdraw from class: Tuesday, October 22. &\\
Midterm Date and Venue: October 14 (in class). &\\
Final Date: Friday, December 13, 2:00PM-5:00PM. &
\end{tabular}


\section{\sc Textbook}
We will mostly follow \emph{Linear Algebra, 4th Edition}
by Stephen H. Friedberg, Arnold J. Insel, Lawrence E. Spence. Pearson, 2003.

\section{\sc Prerequisites}
A good understanding of 3-dimensional Euclidean geometry. Functions. Polynomial functions in one variable. It helps to have a basic understanding of calculus (or to be taking it concurrently). Since this is a second course on linear algebra with treated rigorously, we will assume some familiarity with basic matrix operations, such as row/column reduction of matrices, and reduced echelon forms.


\section{\sc Content}
Our basic goal is to have a basic understanding of the important subject of linear algebra in a rigorous manner. Linear Algebra is the fundamental mathematical tool you use after simplifying scientific models by their linear approximation. Along the way, you will learn how to articulate your logical deductions. 

 Our plan is to cover the essential part of the seven chapters of the textbook (with the exception of Chapter 3, which is assumed to be familiar with by prerequisite), as well as some additional topics if time permits. Of course we'll adjust the speed of teaching as we proceed along. Here is a list of sections of the textbook that we will cover, which we will be going through at the speed of about 3 sections per week, in the following order:
\begin{list2}
\item Chapter 1, 1.1--1.6.
\item Chapter 2, 2.1--2.5.
\item Chapter 4, 4.1--4.3.
\item Chapter 5, 5.1--5.3.
\item Chapter 7, 7.1--7.3.
\item Chapter 6, 6.1--6.5.
\end{list2}

\section{\sc Advice}
{\bf You are required to attend all the lectures}. Since our
lectures may differ from the text book, and the
schedule might change in occasion. As a general principle for
taking math courses, \emph{take twice the amount of time of lectures
to review what you learnt in class, and do a lot of exercises!} What
we hope to achieve is not only the knowledge but also the ability
to think logically and independently. Feel free to let me know if
some points are unclear to you and ask for more explanations. Any
suggestions about the teaching will be warmly welcomed.

In case you have a conflict with any of the exams, please contact the
instructor as soon as possible and at least two weeks before the
exam. I will schedule a make up exam for you in my office. In the case of a fire alarm or a similar emergency evacuation during either of the midterms or the final exam, leave your exams in the room, face down, before evacuating. Under no circumstances should you take the exam with you.


\section{\sc Assignments}
Homeworks will be assigned each Wednesday, and due the Wednesday a week after. No late homeworks will be accepted. Discussing the problems with other students is encouraged, but each student must write solutions on his/her own. Quizzes and exams will have similar problems as the examples we
do in class and exercises you will do after class. As another general principle in math,
\emph{practice makes perfect}.

\section{\sc Grading}
Your final grade will be based on weekly assignments, a midterm exam and a final exam. We will de-emphasize counting homeworks towards your final grade, since this is the place you are allowed to make mistakes, and get them clarified.
\\

\begin{list2}
\item Homeworks, 10 \%
\item Quizzes, 20 \%
\item Mid-term, 30 \%
\item Final, 40 \%
\end{list2}

\section{\sc Academic Honesty} 
Honor is a core value of the University of Virginia, an integral part of its educational mission, and the
foundation of the student experience. As such, the Honor System applies both in the classroom and
beyond.

An Honor Offense is defined as any Act of Lying, Cheating, or Stealing, where such act was committed with Knowledge, and is Significant.


Students found guilty of an Honor offense are permanently dismissed from the University, and those who have graduated from the University are subject to degree revocation by the General Faculty. 
\end{resume}

\end{document}



