\documentclass[12pt]{article}
\title{}
\date{}
\usepackage{amsmath,amsfonts,amsthm,amssymb,mathrsfs,xypic}
\usepackage{times}
%\usepackage{hyperref}
%\usepackage{CJKutf8}
\textwidth=16cm
\textheight=23cm
\voffset=-1.8cm
\hoffset=-1.1cm
%------------------------------------------------------------------------------%
\theoremstyle{plain}
\newtheorem{thm}{Theorem}[section]
\newtheorem{lem}[thm]{Lemma}
\newtheorem{cor}[thm]{Corollary}
\theoremstyle{definition}
\newtheorem{defn}{Definition}[section]
\theoremstyle{remark}
\newtheorem*{remark}{Remark}
\newtheorem*{Ex}{Example}
\newtheorem*{ex}{Exercise}

\newcommand{\ba}{\beta}
\newcommand{\da}{\delta}
\newcommand{\Da}{\Delta}
\newcommand{\ta}{\theta}
\newcommand{\Ta}{\Theta}
\newcommand{\la}{\lambda}
\newcommand{\Oa}{\Omega}
\newcommand{\vphi}{\varphi}

\newcommand{\pl}{\partial}
\newcommand{\fc}{\frac}
\newcommand{\na}{\nabla}
\newcommand{\ol}{\overline}
\newcommand{\lt}{\left}
\newcommand{\rt}{\right}
\newcommand{\rw}{\rightarrow}

\newcommand{\mf}{\mathbf}
\newcommand{\mb}{\mathbb}
\newcommand{\ml}{\mathcal}

\newcommand{\bs}{\boldsymbol}
\newcommand{\wt}{\widetilde}

\pagestyle{empty}

\begin{document}
\begin{center}
{\Large \bf Exercises for Week 2}
\end{center}
The work handed in should be entirely your own. You can consult Dummit and Foote, Artin and/or the class notes but nothing else. To receive full credit, justify your answer in a clear and logical way. Due Sept. 18.

\begin{enumerate}
\item Show by example for subgroups in $S_3$ that, if $H$ and $K$ are subgroups, then $H\cup K$ is not necessarily a subgroup.

\item Show that if $(A,+)$ is an abelian group and $H$, $K$ are subgroups, then 
$$H+K:=\{x\in A|x=h+k~ \textrm{for some $h\in H$, $k\in K$}\}$$ is a subgroup, and it is the \emph{smallest} subgroup that contains both $H$ and $K$.

\item 
\begin{enumerate}
\item Prove that, if $(G,\star_G)$, $(H,\star_H)$ are given two groups, then $G\times H$ with the product structure $\star$ defined by
$$(g_1,h_1)\star (g_2,h_2):=(g_1\star_G g_2, h_1 \star h_2)$$
for any $(g_1,h_1),(g_2,h_2)\in G\times H$ is a group. This is called the \emph{product group} of $G$ and $H$. 
\item Show that $G\times \{e_H\}$ is a subgroup of $G\times H$.
\item Consider $G=H=(\mathbb{Z},+)$. Construct five different subgroups of $\mathbb{Z}\times \mathbb{Z}$.
\end{enumerate}

\item Perform the Euclidean algorithm to find the $\gcd$ of

(1). $90$ and $300$.

(2). $60$ and $17$.

\item Consider the set $H$ of matrices
$$
\left\{
\left(
\begin{tabular}{cc}
$1$ & $a$\\$0$ & $1$
\end{tabular}
\right)
|a\in \mathbb{R}
\right\}\subset M_2(\mathbb{R})
$$
under the usual matrix multiplication. 
\begin{enumerate}
\item Prove that it is a subgroup of $GL(2,\mathbb{R})$. Is it abelian? Can you identify it with a more familiar group?
\item Find a countably infinite subgroup of $H$.
\end{enumerate}
\item Consider the maps\\
 (1) $f: \mathbb{R}^2\longrightarrow \mathbb{R}, (x,y)\mapsto x^2+y^2$.\\
 (2) $f: \mathbb{R}\longrightarrow \mathbb{R}, x\mapsto x^3$\\
 (3) $f: \mathbb{C}\longrightarrow \mathbb{C}, z\mapsto z^3$\\
Determine the equivalence relation on the respective domains determined by $f$. Namely, explicitly describe equivalence classes coming from the map $f$.

\item Show that if $a=a_n10^n+a_{n-1}10^{n-1}+\dots+a_110+a_0$, where each $a_i\in \{0,1,2,\dots, 9\}$. Then
\[a\equiv a_n+a_{n-1}+\dots+ a_1+a_0\quad (\mathrm{mod}{9}). \]

\item Use the theorem on classification of subgroups of $\mathbb{Z}$ to prove that, if $a_1, a_2,\dots , a_n \in \mathbb{Z}$, then
\[\gcd(a_1,a_2,\dots, a_n)=\gcd(\gcd(a_1,\dots, a_k),\gcd(a_{k+1},\dots, a_n))\]
for any $1\leq k \leq n$.
\end{enumerate}
\end{document} 