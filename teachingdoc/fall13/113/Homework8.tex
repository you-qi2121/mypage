\documentclass[12pt]{article}
\title{}
\date{}
\usepackage{amsmath,amsfonts,amsthm,amssymb,mathrsfs}
\usepackage{times}
\usepackage{diagcat}
\usepackage{hyperref}
%\usepackage{hyperref}
%\usepackage{CJKutf8}
\textwidth=16cm
\textheight=23cm
\voffset=-1.8cm
\hoffset=-1.1cm
%------------------------------------------------------------------------------%
\theoremstyle{plain}
\newtheorem{thm}{Theorem}[section]
\newtheorem{lem}[thm]{Lemma}
\newtheorem{cor}[thm]{Corollary}
\theoremstyle{definition}
\newtheorem{defn}{Definition}[section]
\theoremstyle{remark}
\newtheorem*{remark}{Remark}
\newtheorem*{Ex}{Example}
\newtheorem*{ex}{Exercise}

\newcommand{\ba}{\beta}
\newcommand{\da}{\delta}
\newcommand{\Da}{\Delta}
\newcommand{\ta}{\theta}
\newcommand{\Ta}{\Theta}
\newcommand{\la}{\lambda}
\newcommand{\Oa}{\Omega}
\newcommand{\vphi}{\varphi}

\newcommand{\pl}{\partial}
\newcommand{\fc}{\frac}
\newcommand{\na}{\nabla}
\newcommand{\ol}{\overline}
\newcommand{\lt}{\left}
\newcommand{\rt}{\right}
\newcommand{\rw}{\rightarrow}

\newcommand{\mf}{\mathbf}
\newcommand{\mb}{\mathbb}
\newcommand{\ml}{\mathcal}

\newcommand{\bs}{\boldsymbol}
\newcommand{\wt}{\widetilde}

\pagestyle{empty}

\begin{document}
\begin{center}
{\Large \bf Exercises for Week 8}
\end{center}
The work handed in should be entirely your own. You can consult Dummit and Foote, Artin and/or the class notes but nothing else. To receive full credit, justify your answer in a clear and logical way. Due Oct. 30.

\paragraph{Reading.}
\begin{itemize}
\item With Dummit and Foote, please read Section 6.3.  Alternatively, read Artin Sections 7.9, 7.10.
\item {\bf Important}! Make sure you are comfortable with linear algebra. Read through Artin Sections 4.1-4.4, 5.1 and try some exercises from these sections. Alternatively, revisit your old linear algebra textbook on the corresponding material.
\end{itemize}

\begin{enumerate}
\item Let $G$ and $H$ be two groups. Show that if $G$ can be generated by $n$ elements, $H$ can be generated by $m$ elements, then $G\times H$ can be generated by $n+m$ elements.
\item In class, we have shown that picking two elements in a group $G$ is equivalent to defining a homomorphism $\phi$ from the free group on two letters $F\{x,y\}$ to $G$. Now let $G=U(1):=\{e^{i\theta}|\theta\in [0,2\pi)\}$, and pick the elements $x=e^{2\pi i/3}$ and $y=e^{2\pi i /4}$. Determine what a general element $x^{a_1}y^{b_1}x^{a_2}y^{b_2}\dots x^{a_r}y^{b_r}$ ($a_i,b_i\in \mathbb{Z}\backslash \{0\}$, $b_r$ may be zero) is mapped to under $\phi$. Determine the size of the image group $\mathrm{Im}(\phi)$.
\item Show that the (lattice) Heisenberg group
\[
H:=\left\{\left(
\begin{array}{ccc}
1 & a & b\\
0 & 1 & c\\
0 & 0 & 1
\end{array}
\right)\Bigg|a,b,c\in \mathbb{Z}\right\}
\]
has a group presentation as $\langle p,q,z|pz=zp, qz=zq, pqp^{-1}q^{-1}=z\rangle$. (Hint: Find explicitly what matrices $p,~q,~z$ should be.)
\item Present the groups\\
(a) $\mathbb{Z}\times \mathbb{Z}$, \\
(b) $\mathbb{Z}/(5)\times \mathbb{Z}/(7)$ \\
by two generators and relations.
\item In class we introduced the braid group on $n$ strands. Let $n=3$. Construct five elements in terms of the braid generators $\sigma_1$ and $\sigma_2$ that are in the kernel of the homomorphism
    \[Br_3\mapsto S_3,~\sigma_i\mapsto (i,i+1), \quad(i=1,2).\]
Try to give your answer in terms of braid pictures.
For more introductory information on the braid group, watch the nice Youtube video:\\
\href{https://www.youtube.com/watch?v=u3Gt578803I}{https://www.youtube.com/watch?v=u3Gt578803I}
 \end{enumerate}
\end{document} 