\documentclass[12pt]{article}
\title{}
\date{}
\usepackage{amsmath,amsfonts,amsthm,amssymb,mathrsfs}
\usepackage{times}
\usepackage{diagcat}
%\usepackage{hyperref}
%\usepackage{CJKutf8}
\textwidth=16cm
\textheight=23cm
\voffset=-1.8cm
\hoffset=-1.1cm
%------------------------------------------------------------------------------%
\theoremstyle{plain}
\newtheorem{thm}{Theorem}[section]
\newtheorem{lem}[thm]{Lemma}
\newtheorem{cor}[thm]{Corollary}
\theoremstyle{definition}
\newtheorem{defn}{Definition}[section]
\theoremstyle{remark}
\newtheorem*{remark}{Remark}
\newtheorem*{Ex}{Example}
\newtheorem*{ex}{Exercise}

\newcommand{\ba}{\beta}
\newcommand{\da}{\delta}
\newcommand{\Da}{\Delta}
\newcommand{\ta}{\theta}
\newcommand{\Ta}{\Theta}
\newcommand{\la}{\lambda}
\newcommand{\Oa}{\Omega}
\newcommand{\vphi}{\varphi}

\newcommand{\pl}{\partial}
\newcommand{\fc}{\frac}
\newcommand{\na}{\nabla}
\newcommand{\ol}{\overline}
\newcommand{\lt}{\left}
\newcommand{\rt}{\right}
\newcommand{\rw}{\rightarrow}

\newcommand{\mf}{\mathbf}
\newcommand{\mb}{\mathbb}
\newcommand{\ml}{\mathcal}

\newcommand{\bs}{\boldsymbol}
\newcommand{\wt}{\widetilde}

\pagestyle{empty}

\begin{document}
\begin{center}
{\Large \bf Exercises for Week 4}
\end{center}
The work handed in should be entirely your own. You can consult Dummit and Foote, Artin and/or the class notes but nothing else. To receive full credit, justify your answer in a clear and logical way. Due Oct. 2.

\paragraph{Reading.} With Dummit and Foote, please read Section 1.6, 3.1, 3.2. Alternatively, read Artin 2.8, 2.5.

\begin{enumerate}
\item If $G$ is a finite group, use Lagrange's theorem $|G|=[G:K]|K|$ to show the more general version: if $K\subset H \subset G$ are inclusions of subgroups, then $[G:K]=[G:H][H:K]$.
\item Prove that if $H$, $K$ are finite subgroups of a group $G$ whose orders are coprime, then $K\cap H=\{1_G\}$.
\item Find the minimal coset representative in $S_5/S_4$ of the element
\[
\begin{DGCpicture}[scale=0.6]
\DGCstrand(0,0)(4,2)
\DGCstrand(1,0)(4,1.35)(3,2)
\DGCstrand(2,0)(0,2)
\DGCstrand(3,0)(1,2)
\DGCstrand(4,0)(2,2)
\end{DGCpicture} \ .
\]
Here by ``minimal'' we mean it should have the minimal number of crossings in its left coset, as we did in class. (Hint: You might find it useful that, locally, you can wiggle pictures using the relation $
\begin{DGCpicture}[scale=0.5]
\DGCstrand(0,0)(2,2)
\DGCstrand(2,0)(0,2)
\DGCstrand(1,0)(0,1)(1,2)
\end{DGCpicture}
\ = \
\begin{DGCpicture}[scale=0.5]
\DGCstrand(0,0)(2,2)
\DGCstrand(2,0)(0,2)
\DGCstrand(1,0)(2,1)(1,2)
\end{DGCpicture}
$ )

\item Recall we have shown in class that $\mathbb{Z}/(p)^*:=\mathbb{Z}/(p)\backslash\{0\}$ is an abelian group of order $p-1$ under multiplication of modular numbers. Use this fact to show Fermat's little theorem: for any integer $a\in \mathbb{Z}$, $a^p\equiv a~(\mathrm{mod}~p)$. (Hint: check that if $(a,p)=1$, $a^{p-1}\equiv 1~(\mathrm{mod}~p)$. To do this remember what we have said about orders of elements in a finite group).

\item In class we discussed that picking any element $x\in G$ uniquely determines a group homomorphism $\mathbb{Z}\longrightarrow G$. Consider in $SO(2,\mathbb{R})$, and pick
    $$x=
    \left(
    \begin{array}{cc}
    \cos \theta &-\sin \theta\\
    \sin \theta & \cos \theta
    \end{array}
    \right)
    $$
   Here $\theta$ is a fixed angle. Write down explicitly the group homomorphism determined by this element. When is this homomorphism injective?
\item Let $K$ be a subgroup of a given group $G$.\\
(a) Show that for a fixed $g\in G$, $gKg^{-1}:=\{gkg^{-1}|k\in K\}$ is a subgroup of $G$.\\
(b) Show that $|K|=|gKg^{-1}|$.\\
(c) Show that if $G$ has only one subgroup $K$ of order $n$, then $K$ is normal.
\item Determine under what condition the inverse map $(-)^{-1}:G\longrightarrow G, g\mapsto g^{-1}$ is a group homomorphism.
\end{enumerate}
\end{document} 