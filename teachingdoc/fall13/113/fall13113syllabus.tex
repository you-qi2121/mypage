\documentclass[margin,line]{res}
\usepackage{url}

\oddsidemargin -.5in \evensidemargin -.5in \textwidth=6.0in
\itemsep=0in
\parsep=0in

\newenvironment{list1}{
  \begin{list}{\ding{113}}{%
      \setlength{\itemsep}{0in}
      \setlength{\parsep}{0in} \setlength{\parskip}{0in}
      \setlength{\topsep}{0in} \setlength{\partopsep}{0in}
      \setlength{\leftmargin}{0.17in}}}{\end{list}}
\newenvironment{list2}{
  \begin{list}{$\bullet$}{%
      \setlength{\itemsep}{0in}
      \setlength{\parsep}{0in} \setlength{\parskip}{0in}
      \setlength{\topsep}{0in} \setlength{\partopsep}{0in}
      \setlength{\leftmargin}{0.2in}}}{\end{list}}


\begin{document}

\name{Syllabus of Introduction to Abstract Algebra \vspace*{.1in}}

\begin{resume}

\section{\sc Related Information}
\vspace{.05in}
\begin{tabular}{@{}p{3.4in}p{4in}}
 Instructor: You Qi            &   \\
Time: Every Mon. Wed. Fri. 9:00--10:00am & Venue: 102 Wurster\\
Office: Room 849, Evans   & E-mail:  yq2121@berkeley.edu \\
Office Hour: Mon. Wed. Fri. 10:30--11:30am & \\
\end{tabular}
Course webpage: \url{http://math.berkeley.edu/~yq2121/fall2013113.html}


\section{\sc Textbook}
We will not have a fixed book. But if you feel like buying one for collection, I would recommend
\emph{Abstract Algebra, 3rd Edition}
by D.~S.~Dummit and R.~M.~Foote, which contains a lot more material than that can be covered in a single course (which, of course, may be useful for research in the future if you are into that kind of nerdy stuff...). Another great reference is M.~Artin's algebra book, which is less dry than the above but contains less material.

\section{\sc Schedule}
Our basic goal is to cover some basic group theory, ring and module theory. This might be too ambitious a goal, but I believe this is something useful to learn. Of course we'll adjust the speed of teaching as we proceed along. Here is a tentative plan of the staff we'll cover.
\begin{enumerate}
\item Basic concepts of group theory: Associative laws. Groups, subgroups. Examples of groups. Equivalence relations. Cosets. Congruence numbers and cyclic groups. Group homomorphisms. Normal subgroups. Isomorphisms. Quotient groups. The correspondence theorem. Free groups. Generators and relations.
\item Group actions and geometry: Isometries. Isometries of two-dimensional real space. Group actions on sets. Cayley's theorem. The counting formula. Actions on cosets. Class equations. Discrete subgroups of isometries. Dihedral groups. Finite subgroups of SO(3).

\item Basic ring theory: Definition of a ring. Polynomial rings. Maps of rings and ideals. Quotient rings. Product rings. Principle ideal domains.
\item Modules: Definitions. Submodules. Maps of modules. Quotient modules. Free modules and presentations. Applications of modules over PID. Introduction to representation theory.
%\begin{list2}
%\item Jan. 17, \S 1.1-\S 1.2,
%\item Jan. 19, \S 1.3-\S 1.5,
%\item Jan. 24, \S 1.6-\S 1.8,
%\item Jan. 26, \S 1.10-\S 1.11, Quiz 1.
%\end{list2}
%\item Chapter 2 Functions (3 lectures).
%\begin{list2}
%\item Jan. 31, \S 2.1- \S 2.3,
%\item Feb. 2, \S 2.4- \S 2.5,
%\item Feb. 7, \S 2.6-\S 2.7, Quiz 2.
%\end{list2}
%\item Chapter 3 Polynomials and Rational Functions (4 lectures).
%\begin{list2}
%\item Feb. 9, \S 3.1-\S 3.2,
%\item Feb. 14, \S 3.3-\S 3.4,
%\item Feb. 16, \S 3.5-\S 3.6,
%\item Feb. 21, \S 3.7, Quiz 3.
%\end{list2}
%\item Chapter 4 Exponentials and Logarithmic Functions (3 lectures).
%\begin{list2}
%\item Feb. 23, \S 4.1-\S 4.2,
%\item Feb. 28, \S 4.3-\S 4.4,
%\item Mar. 1, \S 4.5-\S 4.6, Quiz 4.
%\end{list2}
%\item Mar. 6, Review session.
%\item Mar. 8, {\bf Mid-term}.
%\item Mar. 12-16, Spring break.
%\item Chapters 5-6 Trigonometric Functions (6 lectures)
%\begin{list2}
%\item Mar. 20, \S 5.1-\S 5.2,
%\item Mar. 22, \S 5.3-\S 5.4,
%\item Mar. 27, \S 5.5 \S 6.1,
%\item Mar. 29, \S 6.2-\S 6.3,
%\item Apr. 3, \S 6.4-\S 6.5,
%\item Apr. 5, \S 6.6, Quiz 5.
%\end{list2}
%\item Chapter 7 Analytic Geometry (3 lectures)
%\begin{list2}
%\item Apr. 10, \S 7.1-\S 7.2,
%\item Apr. 12, \S 7.3-\S 7.4,
%\item Apr. 17, \S 7.5, Quiz 6.
%\end{list2}
%\item Chapter 8 Polar Coordinates and Parametric Equations (2 lectures)
%\begin{itemize}
%\item Apr. 24, \S 8.1-\S 8.2,
%\item Apr. 26, \S 8.3-\S 8.4, Quiz 7.
%\end{itemize}
%\item Review session TBA.
%\item {\bf Final} TBA.
\end{enumerate}

\section{\sc Advice}
{\bf You are required to attend all the lectures}. Since our
lectures do not follow any particular text book, and the
schedule might change in occasion. As a general principle for
taking math courses, \emph{take twice the amount of time of lectures
to review what you learnt in class, and do a lot of exercises!} What
we hope to achieve is not only the knowledge but also the ability
to think logically and independently. Feel free to let me know if
some points are unclear to you and ask for more explanations. Any
suggestions about the teaching will be warmly welcomed.

In case you have a conflict with any of the exams, please contact the
instructor as soon as possible and at least two weeks before the
exam. I will schedule a make up exam for you in my office.


\section{\sc Assignments}
Homeworks will be assigned every week, and should
be turned in on Wednesdays before classes. No late homeworks will be considered.
Quizzes and exams will be similar to the examples we
do in class and exercises. As another general principle in math,
\emph{practice makes perfect}.

\section{\sc Grading}
\begin{list2}
\item Homeworks 5 \%
\item Quizzes, 20 \%
\item Mid-term, 35 \%
\item Final, 40 \%
\end{list2}

%\section{\sc Academic Honesty} Any form of
%cheating during midterm or final will result in you failing the
%course and the matter being reported to your dean.

\end{resume}

\end{document}



