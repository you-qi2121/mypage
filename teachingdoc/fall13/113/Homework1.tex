\documentclass[12pt]{article}
\title{}
\date{}
\usepackage{amsmath,amsfonts,amsthm,amssymb,mathrsfs}
\usepackage{times}
%\usepackage{hyperref}
%\usepackage{CJKutf8}
\textwidth=16cm
\textheight=23cm
\voffset=-1.8cm
\hoffset=-1.1cm
%------------------------------------------------------------------------------%
\theoremstyle{plain}
\newtheorem{thm}{Theorem}[section]
\newtheorem{lem}[thm]{Lemma}
\newtheorem{cor}[thm]{Corollary}
\theoremstyle{definition}
\newtheorem{defn}{Definition}[section]
\theoremstyle{remark}
\newtheorem*{remark}{Remark}
\newtheorem*{Ex}{Example}
\newtheorem*{ex}{Exercise}

\newcommand{\ba}{\beta}
\newcommand{\da}{\delta}
\newcommand{\Da}{\Delta}
\newcommand{\ta}{\theta}
\newcommand{\Ta}{\Theta}
\newcommand{\la}{\lambda}
\newcommand{\Oa}{\Omega}
\newcommand{\vphi}{\varphi}

\newcommand{\dif}{\partial}
\newcommand{\fc}{\frac}
\newcommand{\na}{\nabla}
\newcommand{\ol}{\overline}
\newcommand{\lt}{\left}
\newcommand{\rt}{\right}
\newcommand{\rw}{\rightarrow}
\newcommand{\lra}{\longrightarrow}

\newcommand{\mf}{\mathbf}
\newcommand{\mb}{\mathbb}
\newcommand{\ml}{\mathcal}

\newcommand{\bs}{\boldsymbol}
\newcommand{\wt}{\widetilde}

\pagestyle{empty}

\begin{document}
\begin{center}
{\Large \bf Exercises for Week 1}
\end{center}
The work handed in should be entirely your own. You can consult Dummit and Foote, Artin and/or the class notes but nothing else. To receive full credit, justify your answer in a clear and logical way. Due Sept. 11.

\paragraph{Reading.} If you have Dummit and Foote, read Sections 1.1 and 2.1. With Artin, read Sections 2.1, 2.2.

\begin{enumerate}
\item Determine if the matrix multiplication structure on $\mathrm{M}(n,\mathbb{R})$ is associative/commutative. Is it a group under multiplication, why? (Caution: what's the difference when $n=1$ and $n\neq 1$?)

\item Show that given any set $S$ and a fixed element $x_0\in S$, the map $(x,y)\mapsto x_0$, $\forall x, y \in S$ defines a commutative, associative law of composition.

\item Let $G:=\{a+b\sqrt{2}|(a,b)\in \mathbb{Q}^2\backslash (0,0)\}$. Show that $G$ is a group under multiplication.

\item Prove that if all elements of a group $G$ satisfy $x^2=1$, then $G$ is abelian.

\item Show by definition that $SL(2,\mathbb{R})$ is a subgroup of $GL(2,\mathbb{R})$.

\item List all subgroups of $S_3$.

\item Let $G$ be a group and $g_0$ be a fixed element of $G$. Show that the set $Z_G(g_0):=\{g\in G|gg_0g^{-1}=g_0\}$ is a subgroup of $G$. This is called the \emph{centralizer} of $g_0$ in $G$.
%\item Evaluate $\int_{\frac{2\pi}{3}}^{\pi} \cos x (1+ \sin^2 x) dx.$
%\item Let $m,n$ be positive integers. Use Formula 2 on page 476 to show the following identities.
%	\begin{enumerate}
%	\item  \begin{align*} \int_{-\pi}^\pi \sin mx \cos nx \,\,dx =0 \end{align*}
%	\item
%	\begin{align*}
%	\int_{-\pi}^\pi \cos mx \cos nx \,\,dx = \left\{ \begin{array}{lcc} 0 &if& m \neq n\\ \pi&if& m=n \end{array}\right.
%	\end{align*}
%	\end{enumerate}	
%\item Let $n$ be a positive integer. Let $I_n = \int_0^{\pi/2} \sin^nx \,\,dx.$
%	\begin{enumerate}
%	\item Use integration by parts to show that
%\begin{align*}
%I_n = \frac{n-1}{n} I_{n-2} \end{align*} for $n \geq 2.$
%	\item Show that \begin{align*}
%	I_{2n+1}= \frac{2 \cdot 4 \cdot 6 \cdot \cdots \cdot 2n}{3 \cdot 5 \cdot 7 \cdot \cdots \cdot (2n+1)}
%	\end{align*}
%	\item Show that \begin{align*}
%	I_{2n} = \frac{1 \cdot 3 \cdot 5 \cdot \cdots \cdot (2n-1)}{2 \cdot 4 \cdot 6 \cdot \cdots \cdot 2n} \frac{\pi}{2}
%	\end{align*}
%	\item Show that $I_{2n+2} \leq I_{2n+1} \leq I_{2n}.$ (Hint: $|\sin x| \leq 1.$)
%	\item Show that $\frac{I_{2n+2}}{I_{2n}} = \frac{2n+1}{2n+2}$, $\frac{2n+1}{2n+2} \leq \frac{I_{2n+1}}{I_{2n}} \leq 1$, and $\lim_{n \rw \infty} \frac{I_{2n+1}}{I_{2n}} =1$
%	\item Show the {\it Wallis formula}:
%	\begin{align*}
%	\lim_{n \rw \infty} \frac{2}{1} \cdot \frac{2}{3} \cdot \frac{4}{3} \cdot \frac{4}{5} \cdot \frac{6}{5} \cdot \frac{6}{7} \cdot \cdots \cdot \frac{2n}{2n-1} \cdot \frac{2n}{2n+1} = \frac{\pi}{2}.
%\end{align*}		\end{enumerate}
%For (b) and (c), you need {\it mathematical induction}. Search "Khan academy mathematical induction" for an explanation and read Example 5 in Appendix E(page A35) in Stewart's Calculus for how to write a proof.
\end{enumerate}
\end{document} 