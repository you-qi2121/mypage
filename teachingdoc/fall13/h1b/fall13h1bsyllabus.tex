\documentclass[margin,line]{res}
\usepackage{url}

\oddsidemargin -.5in \evensidemargin -.5in \textwidth=6.0in
\itemsep=0in
\parsep=0in

\newenvironment{list1}{
  \begin{list}{\ding{113}}{%
      \setlength{\itemsep}{0in}
      \setlength{\parsep}{0in} \setlength{\parskip}{0in}
      \setlength{\topsep}{0in} \setlength{\partopsep}{0in}
      \setlength{\leftmargin}{0.17in}}}{\end{list}}
\newenvironment{list2}{
  \begin{list}{$\bullet$}{%
      \setlength{\itemsep}{0in}
      \setlength{\parsep}{0in} \setlength{\parskip}{0in}
      \setlength{\topsep}{0in} \setlength{\partopsep}{0in}
      \setlength{\leftmargin}{0.2in}}}{\end{list}}


\begin{document}

\name{Syllabus of H1B Honors Calculus \vspace*{.1in}}

\begin{resume}

\section{\sc Related Information}
\vspace{.05in}
\begin{tabular}{@{}p{3.4in}p{4in}}
Instructor: You Qi            &   \\
Time: Every Mon.~Wed.~Fri.~2--3pm & Venue: 87 Evans.\\
Office: Room 849, Evans   & E-mail:  yq2121@berkeley.edu \\
Office Hour: Mon.~Wed.~Fri. 10:30-11:30am &\\
GSI: Doosung Park & E-mail: doosung@berkeley.edu\\
Discussion: Every Mon.~Wed.~Fri.~1--2pm & Venue: 85 Evans.\\
Office: 814 Evans & Office Hour: Tue 1:30--3:30pm\\
\end{tabular}
Course webpage: \url{http://math.berkeley.edu/~yq2121/fall2013h1b.html}


\section{\sc Textbook} \emph{Single Variable Calculus}
by James Stewart, 7th edition, Early Transcendentals for UC Berkeley, Cengage Learning.

\section{\sc Schedule}
Our basic goal is to cover chapters 7, 8, 9, 11, 17 of the text book, if time permits. After each chapter we will have a 20-minute quiz (in class) that tests the basic concepts we introduced in the chapter. The quizzes will be mostly true or false, multiple choices, and simple computations which do not require explanations. There will also be a mid-term and a final which will be much longer than quizzes, and which requires detailed explanation of your answers. The following is a tentative plan of the material we will cover in the course.
\begin{enumerate}
\item Chapter 7 Techniques of Integration: Sections 7.1-7.5, 7.8.
%\begin{list2}
%\item Jan. 17, \S 1.1-\S 1.2,
%\item Jan. 19, \S 1.3-\S 1.5,
%\item Jan. 24, \S 1.6-\S 1.8,
%\item Jan. 26, \S 1.10-\S 1.11, Quiz 1.
%\end{list2}
\item Chapter 8 Further applications of integration: Sections 8.1-8.4.
%\begin{list2}
%\item Jan. 31, \S 2.1- \S 2.3,
%\item Feb. 2, \S 2.4- \S 2.5,
%\item Feb. 7, \S 2.6-\S 2.7, Quiz 2.
%\end{list2}
\item Chapter 9 Differential equations: Sections 9.1-9.3, 9.5.
\item {\bf Mid-term, Oct. 9}
%\begin{list2}
%\item Feb. 9, \S 3.1-\S 3.2,
%\item Feb. 14, \S 3.3-\S 3.4,
%\item Feb. 16, \S 3.5-\S 3.6,
%\item Feb. 21, \S 3.7, Quiz 3.
%\end{list2}
\item Chapter 11 Infinite sequences and series: Sections 11.1-11.4,11.6,11.8-11.10.
%\begin{list2}
%\item Feb. 23, \S 4.1-\S 4.2,
%\item Feb. 28, \S 4.3-\S 4.4,
%\item Mar. 1, \S 4.5-\S 4.6, Quiz 4.
%\end{list2}
%\item Mar. 6, Review session.
\item Chapter 17 Second-order differential equations: Sections 17.1-17.4.
%\begin{list2}
%\item Mar. 20, \S 5.1-\S 5.2,
%\item Mar. 22, \S 5.3-\S 5.4,
%\item Mar. 27, \S 5.5 \S 6.1,
%\item Mar. 29, \S 6.2-\S 6.3,
%\item Apr. 3, \S 6.4-\S 6.5,
%\item Apr. 5, \S 6.6, Quiz 5.
%\end{list2}
%\item Chapter 7 Analytic Geometry (3 lectures)
%\begin{list2}
%\item Apr. 10, \S 7.1-\S 7.2,
%\item Apr. 12, \S 7.3-\S 7.4,
%\item Apr. 17, \S 7.5, Quiz 6.
%\end{list2}
%\item Chapter 8 Polar Coordinates and Parametric Equations (2 lectures)
%\begin{itemize}
%\item Apr. 24, \S 8.1-\S 8.2,
%\item Apr. 26, \S 8.3-\S 8.4, Quiz 7.
%\end{itemize}
\item Review session TBA.
\item {\bf Final December 19.}
\end{enumerate}

\section{\sc Advice}
{\bf You are required to attend all the lectures}, since our
lectures might differ slightly from the text book, and the
schedule might change in occasion. As a general principle for
taking math courses, \emph{take twice the amount of time of lectures
to review what you learnt in class, and do a lot of exercises!} What
we hope to achieve is not only the knowledge but also the ability
to think logically and independently. Feel free to let me know if
some points are unclear to you and ask for more explanations. Any
suggestions about the teaching will be warmly welcomed.

In case you can't make  the exams due to medical conditions, please contact the
instructor as soon as possible and a doctor's note and your dean's approval is needed.
I will schedule a make up exam for you in my office, which will use slightly different
problems from the normal exam.


\section{\sc Assignments}
Homeworks will be assigned in class after each lecture. You should hand them in
every Wednesday before the class. \emph{No late homework will be accepted}.
Your GSI will look at them and keep a record whether you handed in or not,
but will not grade them in detail.
Quizzes and exams will be similar to the examples we
do in class and closely related to homeworks. Hence you are strongly encouraged
to work them through and understand them well by yourself.
As another general principle in math,
\emph{practice makes perfect}.

\section{\sc Grading}
\begin{list2}
\item Homework, 5 \%
\item Quizzes, 25 \%
\item Mid-term, 30 \%
\item Final, 40 \%
\end{list2}

%\section{\sc Academic Honesty} Any form of
%cheating during midterm or final will result in you failing the
%course and the matter being reported to your dean.

\end{resume}

\end{document}



